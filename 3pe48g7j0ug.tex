Internal magnetic fields can prominently affect both the evolution and the final fate of stars. They transport angular momentum, can transport chemical species
and are likely involved in powering some of the most puzzling transients in the universe: Super Luminous Supernovae(SLSNe), long Gamma Ray Bursts (long GRBs) and Fast Radio Bursts (FRBs).
For the first time we are in the position to \textbf{reconstruct the history of internal stellar magnetism, from the main sequence to compact objects}.

To test the presence and impact of strong internal magnetic fields across evolutionary phases,
we propose to extend the technique developed by our team for red giant stars, to bright main sequence stars and WDs.
As discussed in \citet{Cantiello_2016}, the observable impact of strong magnetic fields on g mode pulsations is not limited to the core of red giant stars, and
 g mode pulsations in $\gamma$-Doradus, SPB stars and WDs can be totally suppressed by strong internal magnetic fields (See e.g. Fig.\ref{226017}, right panel ).
Non-pulsating stars within these respective instability strips have been tentatively observed \citep{Balona_2011}, and are good candidates for harboring internal fields.
Kepler did not observe many SPB stars because of the rather high Galactic latitude of the observed field (What about K2?), but the TESS satellite is expect to substantially increase the number of observed SPB and $\gamma$-Dor stars.
Together with GAIA data, it will be possible to determine with higher accuracy the position of these asteroseismic targets with respect to
theoretical instability strips determined with the MESA (stellar evolution) and GYRE (pulsation) codes (See e.g. Fig.\ref{226017}, left panel).
Stars in g-mode instability strips that do not show g-mode pulsations are likely hosting internal magnetic fields with amplitudes larger than the critical
magnetic fields. Our preliminary models show that for SPB stars, this technique is sensitive to magnetic fields with amplitude larger than $10^5$ G,
similar to the amplitude of magnetic fields expected by an equipartition dynamo on the main sequence \citep[See e.g.][]{Featherstone_2009,Augustson_2016}.
One complication is the impact of rapid rotation, which is common in main sequence stars with masses $> 1.5M_\odot$. To address this issue,
we plan to use the Dedalus code to calculate the interaction of gravity waves with realistic magnetic field configurations in spherical geometry \citep[e.g.][]{Braithwaite_2006} 
and adding the effect of rotation. Rotation can substantially alter gravity waves, and their subsequent interaction with magnetic fields. Our simulations will consider axisymmetric background magnetic fields so that the wave problem can be studied self-consistently in just two dimensions ($r$ and $\theta$). 