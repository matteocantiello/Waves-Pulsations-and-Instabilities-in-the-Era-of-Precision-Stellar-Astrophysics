To test the presence and impact of strong internal magnetic fields across evolutionary phases,
we propose to extend our techniques to early-type main sequence stars and white dwarfs (WDs). This way we will be able to \textbf{reconstruct the history of internal stellar magnetism, from the main sequence to compact objects.} As discussed in \citet{Cantiello_2016}, the observable impact of strong magnetic fields on g mode pulsations is not limited to the cores of red giant stars, and g mode pulsations in $\gamma$-Doradus, slowly-pulsating B (SPB), and ZZ Ceti stars can be totally suppressed by strong internal magnetic fields (see e.g. Fig.\ref{226017}, right panel ). The g mode suppression can occur because the horizontal displacement of the plasma associated with an internal gravity wave is opposed by magnetic tension. For a strong enough magnetic field, magnetic tension forces exceed the buoyant restoring forces, converting gravity waves into Alfven waves \cite{lecoanet:17}. This condition sets the critical value of the magnetic field necessary to disrupt a g mode in different stellar environments (for example $B_c \simeq 10^5$ G for red giants and SPB stars). 

Non-pulsating main sequence stars within their respective g mode instability strips have been tentatively observed \citep{Balona_2011}, and are good candidates for harboring internal fields.
However, the sample is very limited because Kepler did not observe many SPB stars, and precise stellar properties were not available for most stars so their position relative to instability strips was uncertain. Fortunately, the TESS satellite will observe more than 4000 SPB stars (Handler, Priv. Comm.) and many $\gamma$-Dor stars. Together with precise distance and temperature measurements from GAIA, it will be possible to determine with high accuracy the position of these asteroseismic targets with respect to theoretical instability strips determined with the MESA (stellar evolution) and GYRE (pulsation) codes (See e.g. Fig.\ref{226017}, left panel). Stars in g-mode instability strips that do not show g-mode pulsations are good candidates to host strong internal magnetic fields. Our preliminary models show that for SPB stars, this technique is sensitive to magnetic fields with amplitude larger than $\sim \! 10^5$ G, similar to the amplitude of magnetic fields expected by an equipartition dynamo on the main sequence \citep[See e.g.][]{Featherstone_2009,Augustson_2016}. Therefore, \textbf{we will determine the incidence of strong internal magnetism on the main sequence for stars of different mass, also placing strong constraints on convective core dynamo theories}. The latter is possible because many stars will also have determined surface rotation velocities, which assuming rigid rotation (but see Sec.~3.2.1) enables to estimate the Rossby number in the convective core of the star.

One complication is the impact on wave propagation of rapid rotation, which is common in main sequence stars with masses $M > 1.5M_\odot$. To address this issue,
we plan to use the Dedalus code to calculate the interaction of gravity waves with  magnetic field configurations in spherical geometry \citep[e.g.][]{Braithwaite_2006} 
and adding the effect of rotation. Rotation can substantially alter gravity waves, and their subsequent interaction with magnetic fields. Our simulations will consider axisymmetric background magnetic fields so that the wave problem can be studied self-consistently in just two dimensions ($r$ and $\theta$). 