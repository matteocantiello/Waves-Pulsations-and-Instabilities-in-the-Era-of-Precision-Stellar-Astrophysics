{\color{purple}
\subsubsection{Chemical Mixing by Convectively Excited Waves}
}

Asteroseismology can be used to determine the chemical gradient profile in stars (**cite Conny**). These measurements show the evidence of convective boundary mixing, as well as a smaller amount of chemical mixing within the radiative zone of the star. One possible source of this chemical mixing is transport by waves. **Tami** analyzed tracer particles in a 2D simulation of waves excited by convection, and found that the waves cause the tracer particles to diffuse. The diffusivity is quadratic in the wave amplitude, which varies in the radiative zone due to wave damping and density changes.

\textbf{We will run a series of simplified Dedalus simulations to elucidate the physics of chemical mixing by waves}. First we will run two dimensional simulations in which a single wave mode is excited by hand using a boundary condition. The simulations will solve the wave equations, as well as the advection equation for the chemical species. These simulations will be compared to analytic calculations for the transport of chemical species at second order by the waves. The analytic calculations only need to be carried out to second order as **Tami** shows the diffusion is quadratic in the wave amplitude.

After developing the theory for a single wave, we will run 2D simulations of waves generated by convection, and measure the chemical diffusivity from those simulations. This is similar to the work of **Tami**. However, the goal of these simulations is to confirm that the theory for a single wave can be applied to a spectrum of waves self-consistently generated by convection. If our theory can be applied to a wave spectrum, then we can derive an analytical expression for the che

we will be able to apply our analytic theory of the chemical diffusivity from a single wave to a spectrum of many waves generated by convection to check that the 
