{\color{purple}
\subsubsection{Chemical Mixing by Convectively Excited Waves}
}

Asteroseismology can be used to determine the chemical gradient profile in stars (**cite Conny**). These measurements show the evidence of convective boundary mixing, as well as a smaller amount of chemical mixing within the radiative zone of the star. One possible source of this chemical mixing is transport by waves. **Tami** analyzed tracer particles in a 2D simulation of waves excited by convection, and found that the waves cause the tracer particles to diffuse. The diffusivity is proportional to the square of the wave amplitude, which varies in the radiative zone due to wave damping and density changes.

\textbf{We will run a series of simplified Dedalus simulations to elucidate the physics of chemical mixing by waves}. First we will run two dimensional simulations in which waves are excited by hand using a boundary condition. The simulations will solve the wave equations, as well as the advection equation for the chemical species. These simulations will be compared to analytic calculations for the transport of chemical species at second order by the waves. The analytic calculations only need to be carried out to second order as **Tami** shows the diffusion is quadra
