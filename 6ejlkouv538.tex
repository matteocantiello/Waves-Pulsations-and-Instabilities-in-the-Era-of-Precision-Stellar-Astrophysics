\subsubsection{Baroclinic Instabilities}
\label{baroclinic}


Theories of AM transport are now aided by unprecedented asteroseismic data. Kepler/K2 has recently yielded hundreds of new measurements of internal stellar rotation rates, including intermediate-mass ($1 \, M_\odot \! \lesssim \! M \! \lesssim \! 4 \, M_\odot$) main sequence stars (e.g., \citealt{benomar:15,vanreeth:16}), sub-giants \citep{deheuvels:14}, red giant branch stars \citep{beck:12,mosser:12}, clump/secondary clump stars \citep{mosser:12,deheuvels:15}, and their white dwarf descendants \citep{hermes:17}. Together, these measurements allow us to track internal rotation throughout stellar evolution. However, theoretical explanation of the measurements does not yet exist, as known hydrodynamic and magnetic AM transport mechanisms fail to reproduce observations. They typically predict central rotation rates at least an order of magnitude faster than observed \citep{cantiello:14}, indicating that a more powerful AM transport mechanism is at work. The time is ripe to explore new theories.

A poorly explored but potentially crucial source of AM transport arises from baroclinic instabilities. It is well known that a (radially) differentially rotating star in hydrostatic equilibrium must exhibit a latitudinal gradient in density that embodies a baroclinic state. The horizontal gradient density/temperature contains potential energy capable of driving baroclinic instabilities. Similar instabilities have been studied previously (e.g., \citealt{goldreich:67,knobloch:82,knobloch:83,spruit:83,spruit:84,zahn:93}) with the conclusion that they are non-existent or grow on thermal time scales too slow to produce efficient AM transport. However, Fuller et al. 2018 (in prep) show that non-axisymmetric adiabatic baroclinic instabilies are active and grow on short timescales, in line with the analytic results \citep{tassoul:82,fujimoto:87,fujimoto:88}, simulations \cite{simitev:17}, and numerous geophysical applications (see, e.g., \citealt{pedlosky:92}). The baroclinic instabilities produce nearly horizontal (geostrophic) turbulent motion, which induces a small but crucial vertical diffusion of AM. Fuller et al. 2018 show this diffusivity scales as
\begin{equation}
\label{numri} 
\nu_{\rm AM} = \alpha r^2 \Omega \frac{\Omega^2}{N^2} \, ,
\end{equation}
where $\Omega$ is the rotation rate, $N$ is the Brunt-Vaisala frequency, and $\alpha$ parameterizes the turbulent saturation of the instability. Unlike other instabilities (e.g., GSF instability, ABCD instability, diffusion-aided MRI, and Taylor-Spruit instabilities) is that they are not inhibited by molecular weight gradients present in post-MS stars, and they grow on shearing timescales as opposed to thermal timescales, allowing them to produce significant AM mixing in post-MS stars.




