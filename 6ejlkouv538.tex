{\color{orange}

\subsubsection{Baroclinic Instabilities}
}

Angular momentum (AM) transport within stars remains an essential but outstanding unsolved problem in stellar astrophysics. Fortunately, asteroseismic data provided by Kepler/K2 has recently yielded hundreds of new measurements of internal stellar rotation rates. These measurements have been performed for intermediate-mass ($1 \, M_\odot \! \lesssim \! M \! \lesssim \! 4 \, M_\odot$) main sequence stars (e.g., \citealt{benomar:15,vanreeth:16}), sub-giants \citep{deheuvels:14}, red giant branch stars \citep{beck:12,mosser:12}, clump/secondary clump stars \citep{mosser:12,deheuvels:15}, and their white dwarf descendants \citep{hermes:17}. Together, these measurements allow us to track internal rotation throughout stellar evolutionand build a comprehensive theory of angular momentum transport within stars. However, this theory does not yet exist, as existing hydrodynamic and magnetic angular momentum transport mechanisms fail to reproduce observations. They typically predict central rotation rates at least an order of magnitude faster than observed \citep{cantiello:14,fullerwave:14}, indicating that a more powerful angular momentum transport mechanism is at work. The time is ripe to explore new theories.

A crucial source of angular momentum transport that has not been properly accounted for is baroclinic instability of differentially rotating stars. It is well known that a differentially rotating fluid in hydrostatic equilibrium must exhibit a latitudinal (along surfaces of constant pressure) gradient in density that embodies a baroclinic state. Hence, differentially rotating stars naturally maintain a latitudinal gradient in density, temperature, and entropy that contains potential energy capable of driving baroclinic instabilities. Such instabilities have been studied previously (e.g., \citealt{goldreich:67,knobloch:82,knobloch:83,spruit:84,zahn:93}) with the conclusion that they are non-existent or grow on thermal time scales too slow to produce efficient angular momentum transport. However, Fuller et al. 2018 (in prep) show that baroclinic instabilies are active and grow on short timescales, in line with the analytic results of \citep{tassoul:82,fujimoto:87,fujimoto:88} and simulations of \cite{simitev:17}. This reality is well-known in geophysical applications (see, e.g., \citealt{pedlosky:92}) but appears to have been largely misunderstood in astrophysics. The baroclinic instabilities produce nearly horizontal (geostrophic) turbulent motion, which induces a small but crucial vertical diffusion of angular momentum. Fuller et al. 2018 show this diffusivity scales as
\begin{equation}
\label{numri} 
\nu_{\rm AM} = \alpha r^2 \Omega \frac{\Omega^2}{N^2} \, ,
\end{equation}
where $\Omega$ is the rotation rate, $N$ is the Brunt-Vaisala frequency, and $\alpha$ parameterizes the turbulent saturation of the instability (similar to the $\alpha$ often used to parameterize the magneto-rotational instability). A key difference between baroclinic instabilities and others (such as the GSF, thermal diffusion-aided MRI, and Taylor-Spruit instabilities) is that they are not inhibited by molecular weight gradients present in post-MS stars, and they grow on shearing timescales as opposed to thermal timescales, allowing them to produce significant AM mixing in post-MS stars.




