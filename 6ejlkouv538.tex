{\color{orange}

{\color{orange}

\subsubsection{Baroclinic Instabilities}
}}

AngularAngular momentummomentum (AM)(AM) transport withinwithin starsstars remainsremains anan essentialessential but outstanding unsolved problem in stellar astrophysics. Fortunately, but outstanding unsolved problem in stellar astrophysics. Fortunately, asteroseismic datadata provided by provided by Kepler/K2 has recently yielded hundreds of new measurements of internal stellar rotation rates.rates. TheseThese measurements have been performed for measurements have been performed for intermediate-mass ($1 \, M_\odot \! \lesssim \! M \! \lesssim \! 4 \, M_\odot$) main sequence stars (e.g., \citealt{benomar:15,vanreeth:16}), sub-giants \citep{deheuvels:14}, red giant branch stars \citep{beck:12,mosser:12}, clump/secondary clump stars \citep{mosser:12,deheuvels:15}, and their white dwarf descendants \citep{hermes:17}. Together, these measurements allow us to track internal rotation throughout stellar evolution. However, theoretical explanation of the measurements does not yet exist, as known hydrodynamic and magnetic AM transport mechanisms fail to reproduce observations. They typically predict central rotation rates at least an order of magnitude faster than observed \citep{cantiello:14}, indicating that a more powerful AM transport mechanism is at work. The time is ripe to explore new theories.

A poorly explored but potentially crucial source of AM transport arises from baroclinic instabilitiesinstabilities of differentially rotating stars. of differentially rotating stars. It is well known that a differentially rotating fluidfluid in hydrostatic equilibrium must exhibit a latitudinal (along surfaces of constant pressure) (along surfaces of constant pressure) gradient in density that embodies a baroclinic state. Hence,Hence, differentiallydifferentially rotating stars naturally maintain a latitudinal rotating stars naturally maintain a latitudinal gradient inin density, temperature, and entropy that density, temperature, and entropy that contains potential energy capable of driving baroclinic instabilities. SuchSuch instabilities have been studied previously (e.g., \citealt{goldreich:67,knobloch:82,knobloch:83,spruit:83,spruit:84,zahn:93}) with the conclusion that they are non-existent or grow on thermal time scales too slow to produce efficient AM transport. However, Fuller et al. 2018 (in prep) show that adiabatic baroclinic instabilies are active and grow on short timescales, in line with the analytic results ofof \citep{tassoul:82,fujimoto:87,fujimoto:88} and \citep{tassoul:82,fujimoto:87,fujimoto:88} and simulations ofof \cite{simitev:17}.\cite{simitev:17}. ThisThis reality is well-known in reality is well-known in geophysical applications (see, e.g., \citealt{pedlosky:92})\citealt{pedlosky:92}) but appears to have been largely misunderstood in astrophysics. but appears to have been largely misunderstood in astrophysics. The baroclinic instabilities produce nearly horizontal (geostrophic) turbulent motion, which induces a small but crucial vertical diffusion of AM. Fuller et al. 2018 show this diffusivity scales as
\begin{equation}
\label{numri} 
\nu_{\rm AM} = \alpha r^2 \Omega \frac{\Omega^2}{N^2} \, ,
\end{equation}
where $\Omega$ is the rotation rate, $N$ is the Brunt-Vaisala frequency, and $\alpha$ parameterizes the turbulent saturation of the instabilityinstability (similar(similar toto the $\alpha$ often used to parameterize the magneto-rotational instability). A key difference between adiabatic baroclinic the $\alpha$ often used to parameterize the magneto-rotational instability). A key difference between adiabatic baroclinic instabilities andand others (such as the others (such as the GSF instability, ABCD instability, thermal thermal diffusion-aided MRI, and Taylor-SpruitTaylor-Spruit instabilities)instabilities) isis thatthat theythey are not inhibited by molecular weight gradientsgradients present in post-MS stars, present in post-MS stars, and they grow on shearing timescales as opposed to thermal timescales, allowing them to produce significant AM mixing in post-MS stars.




