To study CBM directly, \citet{Lecoanet_2016a} performed three-dimensional Dedalus simulations, using Cartesion geometry and a convection zone adjacent to a radiative zone (Figure~???). In these simulations, the chemical species abundance evolves according to a diffusion equation, where the diffusivity varied with distance from the boundary of the convection zone. Specifically, the diffusivity decreases like a Gaussian outside the convection zone, i.e., much faster than exponentially. This shows that CBM can accurately modeled as a spatially varying diffusivity, and that this diffusivity can be determined directly from simulations. \textbf{We will run a suite of three-dimensional convection simulations in Dedalus to measure the convective diffusivity as a function of stellar parameters, e.g., size of convection zone, convective Mach number, and properties of the radiative zone.} Using the simulations, we will derive an empirical law for the convective diffusivity profile. We will then implement the diffusivity profiles in MESA, and allow the diffusivity to change as the star evolves.  We will then compute pulsation spectra of these models using GYRE, and compare to the latest TESS asteroseismic observations \citep[similar to][]{moravveji:15,Ghasemi_2016}.

CBM can also be affected by stellar rotation, which can partially inhibit convection by causing the flows to align with the rotation axis \cite[e.g.,][]{Featherstone_2016}, likely weakening the efficiency of CBM. To study this process, we will run spherical simulations of convection in Dedalus, including rotation. It is essential to study the problem in spherical geometry because the effect of rotation is different near the equator than near the pole. Although these simulations are more computationally challenging, we will use the same theoretical techniques to extract a radial