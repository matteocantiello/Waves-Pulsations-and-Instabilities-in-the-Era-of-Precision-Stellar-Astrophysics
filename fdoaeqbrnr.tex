To study CBM directly, \citet{Lecoanet_2016a} evolvedstudied the evolution of a chemical species in three-dimensional Dedalus simulations.  These Cartesian simulations contained a convection zone with an adjacent radiative zone. The horizontal average of the chemical species evolved according to a diffusion equation, where the diffusivity varied with distance from the boundary of the convection zone. Specifically, the diffusivity decreases like a Gaussian outside the convection zone, i.e., much faster than exponentially.

This shows that CBM can accurately modeled as a spatially varying diffusivity, and this diffusivity can be determined directly from three-dimensional simulations. \textbf{We will run a suite of three-dimensional convection simulations in Dedalus to measure the convective diffusivity as a function of stellar parameters, e.g., size of convection zone, convective Mach number, properties of the radiative zone.} The goal of the simulations will be to derive an empirical law for how th

We will implement these radial diffusivity profiles in MESA by interpolating our simulation results to the convective parameters of a stellar model at each stage of its evolution. Thus, the CBM will change self-consistently as the star evolves.

Although the initial simulations will be in Cartesian geometry, the new spherical capabilities of Dedalus will also enable simulations of core convection and convection in shells comparable to the radius of the star. These simulations are computationally more expensive, but are not conceptually difficult as the same analysis strategy can be used to derive radial diffusivity profiles for a wider range of stars.
