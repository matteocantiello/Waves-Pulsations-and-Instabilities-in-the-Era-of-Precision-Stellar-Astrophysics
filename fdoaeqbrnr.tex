To study CBM directly, \citet{Lecoanet_2016a} studied the evolution of a chemical species in three-dimensional Dedalus simulations.  These Cartesian simulations contained a convection zone with an adjacent radiative zone. The horizontal average of the chemical species evolved according to a diffusion equation, where the diffusivity varied with distance from the boundary of the convection zone. Specifically, the diffusivity decreases like a Gaussian outside the convection zone, i.e., much faster than exponentially.

This shows that CBM can accurately modeled as a spatially varying diffusivity, and this diffusivity can be determined directly from three-dimensional simulations. \textbf{We will run a suite of three-dimensional convection simulations in Dedalus to measure the convective diffusivity as a function of stellar parameters, e.g., size of convection zone, convective Mach number, properties of the radiative zone.} Using the simulations, we will derive an empirical law for the convective diffusivity profile. We will implement the diffusivity profiles in MESA, and allow the diffusivity to change as the star evolves. This is in contrast to most previous work in which the diffusivity is assumed to be constant in time. These MESA calculations will be compared to the latest TESS asteroseismic observations \citep[similar to][]{Ghasemi_2016}.

However, we know that CBM is different in stars with similar masses \citep{Stancliffe_2015}. One possible way to break the degeneracy is via rotation. Rotation can partially inhibit convection by causing the flow to partially align with the rotation axis \cite[e.g.,][]{Featherstone_2016}, which would likely weaken the efficiency of CBM. To study this process, we will run spherical simulations of convection in Dedalus, including rotation. It is essential to study the problem in spherical geometry because the effect of rotation is different near the equator than near the pole. Although these simulations are computationally expensive, their analysis is not conceptually difficult as we can use the same strategy to derive radial diffusivity profiles.
