The \citet{Townsend:2017} study reveals a key missing ingredient in the main-sequence evolution of massive stars: \emph{while we have traditionally regarded pulsation as phenomenon that can shed light on the internal structure of these stars, we have overlooked the possibility that pulsation (specifically, heat-driven g modes) may actively be changing this structure}. We propose to address this through a comprehensive series of activities that establish, for the first time, the overall impact of heat-driven modes on massive star evolution. A first step, requiring little in the way of preparatory activities, will be to extend the \citet{Townsend:2017} simulations to the SPBSg stars (the blue region with $M \gtrsim 9\,M_{odot}$ in Figure~\ref{f:xxx}); this will provide a preliminary assessment of whether g modes can be a significant transporter of angular momentum in B supergiants.