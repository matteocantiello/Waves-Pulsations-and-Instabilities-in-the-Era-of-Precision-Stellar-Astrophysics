{\color{red} \subsection{White Dwarfs}}
Many WDs show strong surface magnetic fields ($B>$3MG). The fraction of strongly magnetic WDs is debated, but usually is discussed to be of the order of 10\%. Interestingly, magnetized WDs are systematically more massive \cite{Ferrario_2015}, which could be explained if the observed magnetic fields descend from a main sequence convective core dynamo \cite{Cantiello_2016}. This is because the extension of convective cores during the main sequence extends to mass coordinates above 0.6$M_\odot$ only in stars with initial mass $M>3M_\odot$, so that the entire mass of the WD descendant lies within the mass coordinate occupied by the MS convective core. Ohmic diffusion timescale are larger than the stellar burning timescale and the WD cooling timescale, so magnetic fields are virtually confined in the mass coordinate where they have been produced \cite{Cantiello_2016}. In this scenario, stars with $M>3M_\odot$ produce WDs with $M_{\rm WD}>0.7M_\odot$, similar to the typical masses of magnetic WDs. Moreover, strong magnetic fields may exist within the interior of many WDs, even if not visible at their surface. Magnetic fields in excess of $10^4-10^5 G$ might suppress or modify typical WD g-mode pulsations. Again, observing the absence of pulsations in WDs located in regions predicted to be unstable could help identifying anch

%K2 + Tess + Gaia = hundreds of WDs. Measurements of Rotation using spot modulations