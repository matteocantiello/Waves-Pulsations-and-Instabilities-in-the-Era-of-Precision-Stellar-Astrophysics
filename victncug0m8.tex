
%\vspace{-15pt}
\subsection{Angular Momentum and Chemical Mixing}
%}

\subsubsection{Angular Momentum Transport by Heat-driven Pulsation Modes}

Transport of angular momentum by internal gravity waves (IGWs) has been studied extensively in the context of low-mass stars \citep[e.g.,][]{Schatzman:1993,Kumar:1997,Zahn:1997,Talon:2002,Talon:2005,Rogers:2008aa}, where the waves are excited by turbulent motions within an envelope convection zone. In massive stars, IGWs excited stochastically at the convective core boundary might be important in shaping the internal rotation \citep[e.g.][]{Rogers:2013,Lee:2014,Rogers:2015}. However, heat-driven oscillation modes can also efficiently transport angular momentum, and are  During their main sequence evolution, stars with masses $M \gtrsim 2.5\,M_{\odot}$ pass through one or more phases where pulsation modes (standing acoustic and IGW) are excited by a heat engine mechanism which converts thermal energy into mechanical work (see Figure~\ref{648358})
%The mechanism is similar to the opacity-valve process described by \citet{Eddington:1926} for classical Cepheid pulsators, but keyed to iron and nickel opaci.ty rather than hydrogen and helium opacity.
%It leads to three distinct classes of variable star in the upper part of the main sequence: the slowly pulsating B (SPB) stars ($2.5\,M_{\odot} \lesssim M \lesssim 9\,M_{\odot}$), where g modes (standing IGWs) with periods of days are excited; 
%the $\beta$ Cephei pulsators ($M \gtrsim 9\,M_{\odot$), showing p modes (standing acoustic waves) with periods of hours are seen, and the slowly pulsating B supergiant (SPBSg) stars ($M \gtrsim 15\,M_{\odot}$), showing g modes again with periods of days. 

%Clearly, \emph{every} star above the lower-mass instability cutoff ($M \approx 2.5\,M_{\odot}$) will pass through a phase were g modes are excited. 

Recently, \citet{Townsend:2017} have demonstrated that a single g mode excited to typical observed SPB amplitudes will transport angular momentum so efficiently that the star's internal rotation can be modified on very short timescales. As a specific example, Figure \ref{334438} shows the rotation frequency as a function of radius in an initially non-rotating $4.21,M_{\odot}$ stellar model (blue curve), subject to $10^3$ years of simulated angular momentum transport by a single dipole g mode. Even in this brief time, the mode is able to spin the surface layers up to a significant (and measureable) rotation rate, at the expense of the deeper layers that are spun in the opposite direction to ensure global angular momentum conservation. The resulting rotation profile, with surface and interior counter-rotating, is strikingly different than that produced by standard models for rotational evolution.
%Compare, for instance, against the red curve in the figure, which shows the angular velocity profile of a model set uniformly rotating at the (ZAMS) and allowed to evolve with standard (non-wave) prescriptions for angular momentum transport.


