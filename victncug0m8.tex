
%\vspace{-15pt}
\subsection{Angular Momentum and Chemical Mixing}
%}


{\color{brown}
\subsubsection{Angular Momentum Transport by Heat-driven Pulsation Modes}

Background:
\begin{itemize}
\item Every star with $M \gtrsim 2.5\,\Msun$ passes through one or more heat-driven instability strip during its main sequence evolution
\item In these instability strips, p modes and g modes excited by a kappa mechanism caused by iron opacity
\item g modes thus excited can transport angular momentum from driving regions and deposit it in damping regions (p modes less important)
\item \citet{Townsend:2017} demonstrated how a single g mode can drive the surface layers of slowly-pulsating B stars to $\approx 100\,{\rm km\,s^{-1}}$ on timescales of $10^{3}$ years
\item Thus, heat-driven modes are not just a useful tool for probing the internal structure of massive stars; they can also be a main player in \emph{modifying} this structure.
\end{itemize}

Proposed efforts:
\begin{itemize}
\item Extend angular momentum transport simulations to $M \gtrsim 10\,\Msun$ parts of main sequence, where g modes are excited toward TAMS
\item Also extend transport sims to post-TAMS models (see Convective Boundaries section)
\item Improve GYRE to model effects of differential rotation on wave transport.
\begin{itemize}
\item First step will just use traditional approximation
\item Main effort (and main code development) will involve using a spectral (spherical harmonic) expansion to solve the 2D (r-theta) pulsation problem without need for traditional approximation
\end{itemize}
\item Apply transport sims to specific \emph{Kepler}/\emph{K2} stars with measured multi-mode spectra (i.e., mode parameters, amplitudes), to predict AM fluxes and assess whether they are consistent/contradict current es


}


