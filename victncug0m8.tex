
%\vspace{-15pt}
\subsection{Angular Momentum and Chemical Mixing}
%}



{\color{brown}
\subsubsection{Angular Momentum Transport by Heat-driven Pulsation Modes}

Transport of angular momentum by internal gravity waves (IGWs) has been extensively in the context of lower-mass stars \citep[e.g.,][]{Schatzman:1993,Kumar:1997,Zahn:1997,Talon:2002,Talon:2005,Rogers:2008}, where the waves are excited by turbulent motions within an envelope convection zone. Recent studies explore whether the same waves might be important in shaping the internal rotation of more-massive stars during their main-sequence evolution. The principal focus has been on IGWs excited stochastically at the convective core boundary \citep[e.g.][]{Rogers:2013,Lee:2014,Rogers:2015}. However, there is another potent source of gravity waves which may prove to be even more effective at transporting angular momentum: during their main-sequence evolution, all stars with masses $M \gtrsim 2.5\,M_{\odot}$ pass through one or more phases where pulsation modes (standing acoustic and internal gravity waves) are excited by a heat-engine mechanism which converts thermal energy into mechanical work. The excitation mechanism is similar to that operative in classical Cepheid pulsators, but driven by iron

\citet{Rogers:2013aa} and
\citet{Rogers:2015aa} use two-dimensional anelastic hydrodynamical
simulations to investigate the impact of internal gravity waves
(IGWs), excited at the convective core boundary, on the internal
rotation profile of massive stars. \citet{Lee:2014aa} consider whether
the same IGWs can supply the necessary angular momentum to form the
episodic decretion disk seen in the rapidly rotating Be star HD~51452.


Athough much of the recent focus 

Background:
 \begin{itemize}
 \item Every star with $M \gtrsim 2.5\,M_{\odot}$ passes through one or more heat-driven instability strip during its main sequence evolution
 \item In these instability strips, p modes and g modes excited by a kappa mechanism caused by iron opacity
 \item g modes thus excited can transport angular momentum from driving regions and deposit it in damping regions (p modes less important)
 \item \citet{Townsend:2017aa} demonstrated how a single g mode can drive the surface layers of slowly-pulsating B stars to $\approx 100\,{\rm km\,s^{-1}}$ on timescales of $10^{3}$ years
 \item Thus, heat-driven modes are not just a useful tool for probing the internal structure of massive stars; they can also be a main player in \emph{modifying} this structure.
 \end{itemize}
 
 Proposed efforts:
 \begin{itemize}
 \item Extend angular momentum transport simulations to $M \gtrsim 10\,M_{\odot}$ parts of main sequence, where g modes are excited toward TAMS
 \item Also extend transport sims to post-TAMS models (see Convective Boundaries section)
 \item Improve GYRE to model effects of differential rotation on wave transport.
 \begin{itemize}
 \item First step will just use traditional approximation
 \item Main effort (and main code development) will involve using a spectral (spherical harmonic) expansion to solve the 2D (r-theta) pulsation problem without need for traditional approximation
 \end{itemize}
 \item Apply transport sims to specific \emph{Kepler}/\emph{K2} stars with measured multi-mode spectra (i.e., mode parameters, amplitudes), to predict AM fluxes and assess whether they are consistent/contradict current estimates of internal rotation profile.
 \end{itemize}


    
}




