
%\vspace{-15pt}
\subsection{Angular Momentum and Chemical Mixing}
%}


{\color{brown}
\subsubsection{Angular Momentum Transport by Heat-driven Pulsation Modes}

\begin{itemize}
\item Every star with $M \gtrsim 2.5\,\Msun$ passes through one or more heat-driven instability strip during its main sequence evolution
\item In these instability strips, p modes and g modes excited by a kappa mechanism caused by iron opacity
\item g modes thus excited can transport angular momentum from driving regions and deposit it in damping regions (p modes less important)
\item \citet{Townsend:2017} demonstrated how a single g mode can drive the surface layers of slowly-pulsating B stars to $\approx 100\,{\rm km\,s^{-1}}$ on timescales of $10^{3}$ years
\item Thus, heat-driven modes are not just a useful tool for pro
\end{itemize}


}


