
%\vspace{-15pt}
\subsection{Angular Momentum and Chemical Mixing}
%}

\subsubsection{Angular Momentum Transport by Heat-driven Pulsation Modes}

Transport of angular momentum by internal gravity waves (IGWs) has been extensively in the context of lower-mass stars  \citep[e.g.,][]{Schatzman:1993,Kumar:1997,Zahn:1997,Talon:2002,Talon:2005,Rogers:2008}, where the waves are excited by turbulent motions within an envelope convection zone. Recent studies explore whether the same waves might be important in shaping the internal rotation of more-massive stars during their main-sequence evolution. The principal focus has been on IGWs excited stochastically at the convective core boundary \citep[e.g.][]{Rogers:2013,Lee:2014,Rogers:2015}. However, there is another potent source of gravity waves which may prove to be even more effective at transporting angular momentum: during their main-sequence evolution, stars with masses $M \gtrsim 2.5\,M_{\odot}$ pass through one or more phases where pulsation modes (standing acoustic and internal gravity waves) are excited by a heat-engine mechanism which converts thermal energy into mechanical work. The mechanism is similar to the opacity-valve process described by \citet{Eddington:1926} for classical Cepheid pulsators, but keyed to iron and nickel opacity rather than hydrogen and helium opacity. It leads to three distinct classes of variable star in the upper part of the main sequence: the slowly pulsating B (SPB) stars ($2.5\,M_{\odot} \lesssim M \lesssim 9\,M_{\odot}$), where g modes (standing IGWs) with periods of days are excited; 
%the $\beta$ Cephei pulsators ($M \gtrsim 9\,M_{\odot$), showing p modes (standing acoustic waves) with periods of hours are seen, and the slowly pulsating B supergiant (SPBSg) stars ($M \gtrsim 15\,M_{\odot}$), showing g modes again with periods of days. 


Figure~XXX (Authorea won't insert my figure) shows these distinct classes as instability strips in the Hertzsprung-Russell diagram, as calculated using the MESA and GYRE tools \citep[adapted from][]{Paxton2015}. Clearly, \emph{every} star above the lower-mass instability cutoff ($M \approx 2.5\,M_{\odot}$) will pass through a phase were g modes are excited. Recently, \citet{Townsend:2017} have demonstrated that a single g mode, excited to a moderate amplitude during this phase (where `moderate' is constrained by observations of SPB stars), will transport angular momentum so efficiently that the star's internal rotation can be modified on very short timescales. As a specific example, Figure~YYY shows the equatorial rotation velocity