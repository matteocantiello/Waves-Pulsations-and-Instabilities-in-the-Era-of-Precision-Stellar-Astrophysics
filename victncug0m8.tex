
%\vspace{-15pt}
\subsection{Angular Momentum and Chemical Mixing}
%}

\subsubsection{Angular Momentum Transport by Heat-driven Pulsation Modes}

Transport of angular momentum by internal gravity waves (IGWs) has been extensively in the context of lower-mass stars  \citep[e.g.,][]{Schatzman:1993,Kumar:1997,Zahn:1997,Talon:2002,Talon:2005,Rogers:2008}, where the waves are excited by turbulent motions within an envelope convection zone. Recent studies explore whether the same waves might be important in shaping the internal rotation of more-massive stars during their main-sequence evolution. The principal focus has been on IGWs excited stochastically at the convective core boundary \citep[e.g.][]{Rogers:2013,Lee:2014,Rogers:2015}. However, there is another potent source of gravity waves which may prove to be even more effective at transporting angular momentum: during their main-sequence evolution, stars with masses $M \gtrsim 2.5\,M_{\odot}$ pass through one or more phases where pulsation modes (standing acoustic and internal gravity waves) are excited by a heat-engine mechanism which converts thermal energy into mechanical work. The mechanism is similar to the opacity-valve process described by \citet{Eddington:1926} for classical Cepheid pulsators, but keyed to iron and nickel opacity rather than hydrogen and helium opacity. It leads to three distinct classes of variable star in the upper part of the main sequence: the slowly pulsating B (SPB) stars ($2.5\,M_{\odot} \lesssim M \lesssim 9\,M_{\odot}$), where g modes (standing IGWs) with periods of days are excited; 
%the $\beta$ Cephei pulsators ($M \gtrsim 9\,M_{\odot$), showing p modes (standing acoustic waves) with periods of hours are seen, and the slowly pulsating B supergiant (SPBSg) stars ($M \gtrsim 15\,M_{\odot}$), showing g modes again with periods of days. 


Figure~XXX (Authorea won't insert my figure) shows these distinct classes as instability strips in the Hertzsprung-Russell diagram, as calculated using the MESA and GYRE tools \citep[adapted from][]{Paxton2015}. Clearly, \emph{every} star above the lower-mass instability cutoff ($M \approx 2.5\,M_{\odot}$) will pass through a phase were g modes are excited. Recently, \citet{Townsend:2017} have demonstrated that a single g mode, excited to a moderate amplitude during this phase (where `moderate' is constrained by observations of SPB stars), will transport angular momentum so efficiently that the star's internal rotation can be modified on very short timescales. As a specific example, Figure~YYY shows the linear rotation velocity in the equatorial plane for a $4.21,M_{\odot}$ stellar model (blue curve), which from an initially non-rotating state is subject to 1,000 years of simulated angular momentum transport by a single dipole g mode. Even in this brief time, the mode is able to spin the surface layers up to a significant (and measureable) velocity, at the expense of the deeper layers that are spun in the opposite direction to ensure global angular momentum conservation. The resulting rotation profile, with surface and interior counter-rotating, is strikingly different than that produced by standard models for rotational evolution (shown in the figure by the red curve).

The \citet{Townsend:2017} study reveals a key missing ingredient in the main-sequence evolution of massive stars: \emph{while we have traditionally regarded pulsation as phenomenon that can shed light on the internal structure of these stars, we have overlooked the possibility that pulsation (specifically, heat-driven g modes) may actively be changing this structure}. We propose to address this through a comprehensive series of activities that establish, for the first time, the overall impact of heat-driven modes on massive star evolution. A first step, requiring little in the way of preparatory activities, will be to extend the \citet{Townsend:2017} simulations to the SPBSg stars (the blue region with $M \gtrsim 9\,M_{odot}$ in Figure~\ref{f:xxx}); this will provide a preliminary assessment of whether g modes can be a significant transporter of angular momentum in B supergiants.

The next steps, however, will require a combination of comptational and theoretical efforts, taking place under the guidance of co-PI Townsend with the assistance of graduate student Lopez. The angular momentum transport simulations by \citet{Townsend:2017} do not self-consistently account for the fact that the developing differential rotation profile of the model star will alter the properties (excitation, damping, etc) of the g mode responsible the transport. The GYRE code already has the ability to model this feedback effect, but only approximately through the so-called traditional approximation of rotation (TAR) where some components of the Coriolis force are neglected and the rotation is treated as locally uniform. The advantage of the TAR is that the governing pulsation equations remain separable in all three coordinates, and therefore 