
%\vspace{-15pt}
{\color{orange}
\subsection{Internal Rotation and Angular Momentum Transport}
%}


{\color{brown}
\subsubsection{Angular Momentum Transport by Heat-driven Pulsation Modes}



}





\begin{equation}
\label{numri} 
\nu_{\rm AM} = \alpha r^2 \Omega \frac{\Omega^2}{N^2} \, ,

\end{equation}
where $\Omega$ is the rotation rate, $N$ is the Brunt-Vaisala frequency, and $\alpha$ parameterizes the turbulent saturation of the instability (similar to the $\alpha$ often used to parameterize the magneto-rotational instability). A key difference between baroclinic instabilities and others (such as the GSF, thermal-diffusion aided MRI, and Taylor-Spruit instabilities) is that they are not inhibited by molecular weight gradients common in post-MS stars, and they grow on shearing timescales as opposed to thermal timescales, allowing them to produce more torque in post-MS stars.


}