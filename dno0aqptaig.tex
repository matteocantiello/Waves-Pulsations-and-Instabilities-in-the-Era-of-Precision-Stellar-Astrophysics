\subsection{Internal Magnetic Fields from Asteroseismology}

Magnetic fields can prominently affect both the evolution and the final fate of stars. They can transport both angular momentum and chemical species, and they are likely involved in powering some of the most puzzling transients in the universe: Super Luminous Supernovae(SLSNe), long Gamma Ray Bursts (long GRBs) and Fast Radio Bursts (FRBs). Despite its importance, we know very little about the history of stellar magnetism, because until very recently measurements of stellar magnetism had been limited to surface fields via spectropolarimetry
%and only compact remnants provided some clues about the level of internal magnetization of their progenitor stars.
{\color{green} These observations have revealed that stars with convective envelopes often show the presence of surface magnetic fields that are believed to be produced by contemporary dynamo action. Only about 5\%–10\% of main-sequence (MS) A and OB stars with radiative envelopes exhibit strong (B $\sim$ k G) large-scale magnetic fields, whose origins are debated but may be inherited during the star formation process \citep[fossil fields, see e.g.,][]{2012ASPC..464..405W}}