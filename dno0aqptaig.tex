{\color{red}\subsection{Internal Magnetic Fields from Asteroseismology}}
Magnetic fields can prominently affect both the evolution and the final fate of stars. They transport angular momentum, can transport chemical species
and are likely involved in powering some of the most puzzling transients in the universe: Super Luminous Supernovae(SLSNe), long Gamma Ray Bursts (long GRBs) and Fast Radio Bursts (FRBs). 
Despite its importance, we know very little about the history of stellar magnetism. This is  because, until very recently, 
evidence for stellar magnetism has been limited to surface fields (that can be observed using spectropolarimetry),
and only compact remnants provided some clues about the level of internal magnetization of their progenitor stars.
Spectropolarimetric observations revealed that stars with convective envelopes show the presence of
surface magnetic fields, that are believed to be produced by contemporary dynamo action.
About 5\%–10\% of main-sequence (MS) A and OB stars with radiative envelopes also have strong (B $\sim$ k G)
large-scale magnetic fields that are likely generated or inherited during the star
formation process (fossil fields, see e.g., Aurière et al. 2004; Wade et al. 2012)e{2012ASPC..464..405W} . 