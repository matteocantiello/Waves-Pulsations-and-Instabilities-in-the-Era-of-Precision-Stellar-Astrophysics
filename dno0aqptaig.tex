{\color{red}\subsection{Internal Magnetic Fields from Asteroseismology}}
Magnetic fields can prominently affect both the evolution and the final fate of stars. They transport angular momentum, can transport chemical species
and are likely involved in powering some of the most puzzling transients in the universe: Super Luminous Supernovae(SLSNe), long Gamma Ray Bursts (long GRBs) and Fast Radio Bursts (FRBs). 
Despite its importance, we know very little about the history of stellar magnetism. In particular until very recently
For the first time we are in the position to \textbf{reconstruct the history of internal stellar magnetism, from the main sequence to compact objects}.
Stars with convective envelopes show the presence of
surface magnetic fields. These fields are believed to be produced by contemporary dynamo action.
A small fraction of stars with radiative envelopes also have strong (B $\sim$ k G)
large-scale magnetic fields that are likely generated or inherited during the star
formation process (fossil fields). 
This fraction appears to be small, 5\%–10\% for main-sequence (MS) A and OB stars (e.g., Aurière et al. 2004; Wade et al. 2012). 
Until recently, evidence for stellar magnetism has been limited to surface fields (that can be observed using spectropolarimetry),
and only compact remnants provided some clues about the level of internal magnetization of their progenitor stars.
This has changed thanks to a new asteroseismic technique developed by our team, which allows to detect strong internal magnetic fields.