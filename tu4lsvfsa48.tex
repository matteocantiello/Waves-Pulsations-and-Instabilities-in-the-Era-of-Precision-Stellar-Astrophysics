{\color{brown}
\subsubsection{Predictive Boundaries}

While the standard Schwarzschild and Ledoux criteria for convective (in)stability are well understood, their application to determine the location of convection-zone boundaries entails subtleties that are often overlooked. A specific issue is that elemental abundances at boundaries are typically discontinuous, due to the juxtaposition between well-mixed and unmixed regions. This produces jumps in the radiative ($\nabla_{\rm rad}$) and adiabatic ($\nabla_{\rm ad}$) temperature gradients across the boundary, and an ambiguity in the criterion for neutral stability: should $\nabla_{\rm rad}=\nabla_{\rm ad}$ (in the Schwarzschild case) be applied on the radiative or convective side of the boundary?

\citet{Gabriel:2014} resolved this ambiguity within local mixing length theory (the neutrality criterion must be applied on the convective side, to ensure that convective velocities vanish at the boundary), but implementing this result in stellar evolution codes which do not assume instantaneous mixing is non-trivial. In \citet{Paxton:2017}, co-PI Townsend finally solved the problem with a new \emph{predictive mixing} scheme. This scheme adjusts the di

}