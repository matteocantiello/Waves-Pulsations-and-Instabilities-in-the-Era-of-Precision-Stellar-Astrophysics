{\color{brown}
\subsubsection{Predictive Boundaries}

While the standard Schwarzschild and Ledoux criteria for convective (in)stability are well understood, their application to determine the location of convection-zone boundaries entails subtleties that are often overlooked. A specific issue is that elemental abundances at boundaries are typically discontinuous, due to the juxtaposition between well-mixed and unmixed regions. This produces jumps in the radiative ($\nabla_{\rm rad}$) and adiabatic ($\nabla_{\rm ad}$) temperature gradients across the boundary, and an ambiguity in the criterion for neutral stability: should $\gradr=\grada$ (in the Schwarzschild case) be applied on the radiative or convective side of the boundary?

\citet{Gabriel:2014} resolved this ambiguity 

}