{\color{brown}
\subsubsection{Predictive Boundaries}}


While the standard Schwarzschild and Ledoux criteria for convective (in)stability are well understood, their application to determine the location of convection-zone boundaries entails subtleties that are often overlooked. A specific issue is that elemental abundances at boundaries are typically discontinuous, due to the juxtaposition between well-mixed and unmixed regions. This produces jumps in the radiative ($\nabla_{\rm rad}$) and adiabatic ($\nabla_{\rm ad}$) temperature gradients across the boundary, and an ambiguity in the criterion for neutral stability: should $\nabla_{\rm rad}=\nabla_{\rm ad}$ (in the Schwarzschild case) be applied on the radiative or convective side of the boundary?

\citet{Gabriel:2014} resolved this ambiguity by demonstrating that, within local mixing length theory, the neutrality criterion should be applied on the convective side to ensure that convective velocities vanish at the boundary. Incorporating this behavior in stellar evolution codes has proven non-trivial; but recently Co-PI Townsend implemented a new \emph{predictive mixing} scheme in the MESA code, which modifies mixing diffusivities in the vicinity of convection-zone boundaries to achieve the desired neutrality outcome \citep[see][]{Paxton:2017}. With the new scheme, a number of new phenomena emerge in stellar evolution calculations that motivate confrontation against observations:
\begin{itemize}
\item during the core helium-burning phase (i.e., horizontal branch/red clump), predictive mixing results in core sizes that are maximal; any further extension of the core boundary causes the core to split into two convection zones. These maximal cores are potentially able to explain the large core sizes inferred from \emph{Kepler} measurements of g-mode period spacings, which cannot be explained via standard prescriptions for core boundaries \citep[e.g.,][and reference therein{Constantino:2015}.
\item during the core hydrogen-burning phase (i.e., main sequence) of massive stars, predictive mixing leads to a structure above the convective core consisting of a semi-convection region overlaid by a thin convective shell \citep[see, e.g., the right-hand panel in Fig.~4 of][]{Paxton:2017}. \emph{Kepler} observations of g modes in SPB stars are sensitive to the near-core conditions \cite{Moravveji:2015}

This structure is not seen in simulations without predictive mixing; in fact, in such simulations the region above the core boundary is generally non-converged unless significant convective overshoot is applied.
\item 

They were unable to provide a concrete approach to implementing this behavior in stellar evolution codes that

In \citet{Paxton:2017}, co-PI Townsend developed a new \emph{predictive mixing} scheme in the MESA code; this scheme adaptively modifies the convective mixing diffusivities to 
