{\color{brown}
\subsubsection{Predictive Boundaries}

While the standard Schwarzschild and Ledoux criteria for convective (in)stability are well understood, their application to determine the location of convection-zone boundaries entails subtleties that are often overlooked. A specific issue is that elemental abundances at boundaries are typically discontinuous, due to the juxtaposition between well-mixed and unmixed regions. This produces jumps in the radiative ($\nabla_{\rm rad}$) and adiabatic ($\nabla_{\rm ad}$) temperature gradients across the boundary, and an ambiguity in the criterion for neutral stability: should $\nabla_{\rm rad}=\nabla_{\rm ad}$ (in the Schwarzschild case) be applied on the radiative or convective side of the boundary?

\citet{Gabriel:2014} resolved this ambiguity by demonstrating that, within local mixing length theory, the neutrality criterion should be applied on the convective side to ensure that convective velocities vanish at the boundary. They were unable to provide a concrete approach to implementing this behavior in codes that rely on diffusive

In \citet{Paxton:2017}, co-PI Townsend developed a new \emph{predictive mixing} scheme in the MESA code; this scheme adaptively modifies the convective mixing diffusivites to 

}