{\color{orange}

\section{Relevance to NASA and TCAN}



Our research is highly relevant to NASA's Physics of the Cosmos and Cosmic Origins Programs. Our work will examine the behavior of matter in the extreme conditions present deep inside stars. Importantly, we will make observatinal predictions for observable surface properties of stars that can be used to test the physics operating in the extreme environment of the stellar core. Massive stars and their SNe are important players in the evolution of galaxies and cosmic nucleosynthesis, and revealing the connection between massive stars and SNe is a cornerstone of understanding our Cosmic Origin.

Additionally, our proposed research is closely related to several NASA space missions. First, it complements the SN progenitor detections from {\it HST}. Although {\it HST} will not fly forever, its archival data on SN progenitors will be essential for characterizing progenitor variability of nearby SNe for many decades. Our research will help to establish theoretical understanding of this variability, prolonging {\it HST}'s legacy. Our work will also be useful for interpreting shock-breakout signatures observed by {\it GALEX} and {\it Swift}. Shock-breakout signatures discovered thus far (see Section \ref{supernova}) have revealed larger progenitor radii and shallower density gradients than previously expected, and our work will determine whether wave-driven mass loss can explain these observations.

Similarly, SNe falling within {\it K2} and {\it TESS} fields will have exquisite optical/near-infrared lightcurves that can capture shock breakout signatures and the very early light curve, both of which are shaped by the density profile near the stellar surface at the time of explosion. Our work will determine whether existing and future SN data from {\it K2} and {\it TESS} can be explained as a result of wave-driven mass loss from the progenitor. Finally, exquisite IR data on SNe and their progenitors will be obtained in the near future by {\it JWST}, and further in the future with {\it WFIRST}. Similar to {\it HST}, {\it JWST} may provide exquisite data on SN progenitors through a {\it JWST} LEGUS survey. These data will provide new constraints on progenitor luminosity and variability in the IR, especially for cool RSGs and stars enshrouded in dust due to high pre-SN mass loss rates. Comparison with our work can determine whether wave-driven mass loss occurs in cool/obscured SN progenitors.


}