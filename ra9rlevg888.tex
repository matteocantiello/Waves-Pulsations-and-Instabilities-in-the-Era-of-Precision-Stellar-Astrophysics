
\section{Relavance to TCAN objectives}

\subsection{Relevance to NASA}


Our research is highly relevant to NASA's Physics of the Cosmos and Cosmic Origins Programs. Our work will examine the behavior of matter in the extreme conditions present deep inside stars. Importantly, we will make observational predictions for surface properties of stars that can be used to test the physics operating in the stellar core. Additionally, our proposed research is closely related to several NASA space missions. Most directly, our theoretical work will help maximize the scientific yields from the TESS and K2 missions, allowing for thorough asteroseismic interpretation of TESS data for multiple types of pulsating stars. It will also complement HST and JWST science, e.g., by providing theoretical understanding of overshoot and mixing processes that determine the lifetimes and relative populations (e.g., number of horizontal branch stars) of low-mass stars in clusters. Additionally, our work will be complemented by Gaia data, which combined with TESS data and our theoretical work, will allow for novel studies of populations of pulsating stars (e.g., mapping the SPB instability strip or interpreting pulsating binary stars). This work will also lay new groundwork for asteroseismology from the upcoming WFIRST mission, and subsequently ESA's Plato mission. 


\subsection{Interaction, Collaboration, and Computation}


The work outlined above can only be performed via close collaboration of investigators at different institutions (nodes), making this work ideally suited for TCAN support. Table \ref{841129} lists the personnel, computational tools, and primary scientific objectives of each node. Our goals are highly dependent on the computational tools MESA, GYRE, and Dedalus, which require modest development as discussed above, in accordance with the goals of TCAN.



