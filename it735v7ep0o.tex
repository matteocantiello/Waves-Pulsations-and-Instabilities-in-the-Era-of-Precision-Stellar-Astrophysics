A prime challenge for nonlinear modeling is the complicated interplay of convection and pulsation, as the whole outer envelope is fully convective, with a turnover time similar to the typical pulsation periods.   To address this difficulty, we will extend the GYRE code to include convection-pulsation coupling using the mixing-length perturbative theory described by \citet{Grigahcene:2005}. This will involve modifying the linearized hydrodynamic equations to include momentum terms arising from the divergence of the turbulent pressure tensor, and energy terms arising from the divergence of the convective flux and from kinetic energy dissipation. This work may also begin to better illuminate the connection between pulsations and large mass loss rates highlighted in the original work of \citet{Yoon_2010}. They showed that the amplitudes of pulsations may become large enough to drive a time-dependent 'super-wind' adequate to yield a substantial loss of the outer envelope of the RSG, substantially impacting the appearance of both the progenitor, and its resulting supernova.