 RSG work: 

Heger et al. (1997) explored the onset of the these pulsations using
their linear non-adiabatic code, and explained the instability 
as arising from the traditional Kappa mechanism acting in the
hydrogen ionization zone. They also pursued initial challenging
non-linear calculations of the growth of the instability using a 1D 
hydrodynamic code. They noted that one of the prime challenges to
their ability to diagnose the outcome was the complicated interplay of
convection and pulsation, as the whole outer envelope is fully
convective. We will (RICH!) .... 


followed by.. 


 This work may also begin to better illuminate the connection between
 pulsations and large mass loss rates highlighted in the original work
 of Yoon and Cantiello (2010). They showed that the amplitudes of
 pulsations may become large enough to drive a time-dependent
 'super-wind' adequate to yield a substantial loss of the outer
 envelope of the RSG, substantially impacting how the resulting core
 collapse supernovae would appear to the observer. 