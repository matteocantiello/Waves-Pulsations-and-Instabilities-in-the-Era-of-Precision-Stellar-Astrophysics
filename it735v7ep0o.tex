\citet{1997AampA...327..224H} explored the onset of the these pulsations using
their linear non-adiabatic code, and explained the instability 
as arising from the traditional $\mechanism acting in the
hydrogen ionization zone. They also pursued initial challenging
non-linear calculations of the growth of the instability using a 1D 
hydrodynamic code. They noted that one of the prime challenges to
their ability to diagnose the outcome was the complicated interplay of
convection and pulsation, as the whole outer envelope is fully
convective, with a turnover time similar to or shorter than typical pulsation periods. To address this difficulty, we will extend the GYRE code to include convection-pulsation coupling using the mixing-length perturbative theory described by \citet{Grigahcene:2005}. This will involve modifying the linearized hydrodynamic equations to include momentum terms arising from the divergence of the turbulent pressure tensor, and energy terms arising from the divergence of the convective flux and from kinetic energy dissipation.