\subsection{Effect of Red Supergiant Pulsations on Supernovae}

Given the importance of RSGs to studies of core-collapse supernovae, stellar populations and star formation, we intend to carry out exploratory calculations of a number of aspects of these unusual pulsators. 
%{\color{green} Such modeling can likely impact both the immediate interpretation of observations in distant galaxies and guide the thinking on the likely state of the RSG at the time of final explosion.} 
\citet{1997AampA...327..224H}  explained the instability  as arising from the traditional $\kappa$-mechanism acting in the hydrogen ionization zone. A prime challenge for  modeling is the complicated interplay of convection and pulsation, as the whole outer envelope is fully convective, with a turnover time similar to the typical pulsation periods.   To address this difficulty, we will extend the GYRE code to include convection-pulsation coupling using the mixing-length perturbative theory described by \citet{Grigahcene:2005}. 
%This will involve modifying the linearized hydrodynamic equations to include momentum terms arising from the divergence of the turbulent pressure tensor, and energy terms arising from the divergence of the convective flux and from kinetic energy dissipation. 
In addition, since the dominant modes are often radial (and thus spherical symmetry is maintained), we will explore fully nonlinear hydrodynamic calculations of RSG pulsators with MESA.  This work may also begin to better illuminate the connection between pulsations and large mass loss rates highlighted in the original work of \citet{Yoon_2010}. They showed that the amplitudes of pulsations may become large enough to drive a time-dependent 'super-wind' adequate to yield a substantial loss of the outer envelope of the RSG, substantially impacting the appearance of both the progenitor, and its resulting supernova.


%They also pursued initial non-linear calculations of the instability using a 1D hydrodynamic code, noting 