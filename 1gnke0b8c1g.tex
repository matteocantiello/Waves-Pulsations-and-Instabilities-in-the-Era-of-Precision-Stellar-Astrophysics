\subsection{Effect of Red Supergiant Pulsations on Supernovae}

Given the importance of RSGs to studies of core-collapse supernovae, stellar populations and star formation, we intend to carry out exploratory calculations of a number of aspects of these unusual pulsators. As their dominant pulsations appear to be radial modes, there is opportunity for progress with 1D stellar models such as MESA, both for the linear and non-linear aspects. 
%{\color{green} Such modeling can likely impact both the immediate interpretation of observations in distant galaxies and guide the thinking on the likely state of the RSG at the time of final explosion.} \citet{1997AampA...327..224H}  explained the instability  as arising from the traditional $\kappa$-mechanism acting in the hydrogen ionization zone. 
%They also pursued initial non-linear calculations of the instability using a 1D hydrodynamic code, noting 
A prime challenge for nonlinear modeling is the complicated interplay of convection and pulsation, as the whole outer envelope is fully convective, with a turnover time similar to the typical pulsation periods.