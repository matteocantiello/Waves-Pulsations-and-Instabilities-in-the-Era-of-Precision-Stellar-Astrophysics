{\color{green}

\subsection{Effect of Red Supergiant Pulsations on Supernovae}
}

Prior to core collapse, stars of $M>(10-12) M_\odot$ experience a red
supergiant (RSG) phase, nearly always accompanied by large amplitude
($>0.1$ mag), long-period ($>200 $ day) pulsations.  Given that the
typical massive star spends at most 10$\%$ of it's life as a RSG, and
the rarity of massive stars in general, the historical samples of well
characterized red supergiant pulsators has been small (see Yang \&
Jiang 2012 Ap J 754, 35 for a nice compilation). However, even for
that sample, Yang \& Jiang (2012) showed that pulsating RSGs in the
Galaxy, the Large and Small Magellanic clouds and M33 all followed the
same period luminosity (P-L) relation and that some stars exhibit
oscillation periods consistent with the first overtone radial mode
rather than the more commonly observed fundamental radial mode. The
prime value of these early measurements was to confirm the RSG mass
that is implied by its brightness, a key parameter for studies of star
formation and supernovae.

However, the synoptic sampling enabled by time domain surveys of
nearby galaxies is rapidly changing this situation by dramatically
increasing the available samples. Indeed, Soraisam et al. (2018) just
found 63 pulsating RSGs in M31 using data from five years of
monitoring with the Palomar Transient Factory (PTF). The measured P-L
relation is consistent with that in other galaxies. More importantly,
they reported a clear sequence of likely first-overtone
pulsations. Comparison to stellar evolution models from \texttt{MESA}
confirmed the first overtone hypothesis and indicated that the
variable stars in this sample have $12~M_{\odot}<M<24~M_{\odot}$. The growing prevalence of 
time-domain surveys from the ground, and, studies 
of stellar populations in nearby galaxies from space (HST and JWST) 
will yield even larger samples of this important class of stellar pulsators. 

 Given the importance of RSGs to studies of core-collapse supernovae,
stellar populations and star formation, we intend to carry out
exploratory calculations of a number of aspects of these unusual
pulsators. As their dominant pulsations appear to be the fundamental
radial mode, there is opportunity for progress with 1D stellar models
such as MESA, both for the linear and non-linear aspects. Such
modeling can likely impact both the immediate interpretation of observations in
distant galaxies and guide the thinking on the likely state of the RSG
at the time of final explosion. 

 
   RSG work: 

Heger et al. (1997) explored the onset of the these pulsations using
their linear non-adiabatic code, and explained the instability 
as arising from the traditional Kappa mechanism acting in the
hydrogen ionization zone. They also pursued initial challenging
non-linear calculations of the growth of the instability using a 1D 
hydrodynamic code. They noted that one of the prime challenges to
their ability to diagnose the outcome was the complicated interplay of
convection and pulsation, as the whole outer envelope is fully
convective. We will (RICH!) .... 


followed by.. 


 This work may also begin to better illuminate the connection between
 pulsations and large mass loss rates highlighted in the original work
 of Yoon and Cantiello (2010). They showed that the amplitudes of
 pulsations may become large enough to drive a time-dependent
 'super-wind' adequate to yield a substantial loss of the outer
 envelope of the RSG, substantially impacting how the resulting core
 collapse supernovae would appear to the observer. 