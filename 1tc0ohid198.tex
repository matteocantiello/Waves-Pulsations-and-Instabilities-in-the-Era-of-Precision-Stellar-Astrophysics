{\color{orange}

\subsection{Compact Object Rotation}

The rotation rates of compact objects are determined by the internal physics of their stellar progenitors, and are essential to understanding high energy phenomena such as supernovae and gamma ray bursts. With LIGO's first black hole detections, we now have accurate spin constraints on non-accreting black holes. Combining white dwarf rotation rates from asteroseismology, pulsar rotation rates, and upcoming LIGO black hole detections, we will have a comprehensive picture of internal rotation spanning stellar progenitors of $1 \, M_\odot \lesssim M \lesssim \, 100 \, M_\odot$. Results thus far all point in the same direction: compact object spin rates are usually slow. White dwarfs rotate orders of magnitude slower than breakup (Figure \ref{fig:MRI1p8rot}), while typical newborn pulsars rotate with periods $P_{i} \gtrsim 10-100 \, {\rm ms}$ \citep{faucher:06,igoshev:13,gullon:14} much slower than breakup and too slow to affect the supernova. Initial LIGO results also favor slow black hole birth spins (Figure \ref{BHspin}), as all but one black hole merger is consistent with the aligned component the black hole spins $\chi_

}