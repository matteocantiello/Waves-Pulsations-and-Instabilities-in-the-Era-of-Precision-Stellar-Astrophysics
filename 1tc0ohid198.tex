{\color{orange}

\subsection{Compact Object Rotation}

}

\subsubsection{Baroclinic Instabilities}

The rotation rates of compact objects are determined by the internal physics of their stellar progenitors, and are essential to understanding high energy phenomena such as supernovae and gamma ray bursts. With LIGO's first black hole detections, we now have accurate spin constraints on non-accreting black holes. Combining white dwarf rotation (WD) rates from asteroseismology, neutron star (NS) rotation rates via pulse timing, and upcoming LIGO black hole (BH) detections, we will have a comprehensive picture of internal rotation spanning stellar progenitors of $1 \, M_\odot \! \lesssim \! M \! \lesssim \! 100 \, M_\odot$. All results thus far point in the same direction: compact object spin rates are usually slow, and fast rotation is the exception rather than the rule. WDs rotate orders of magnitude slower than breakup (Figure \ref{fig:MRI1p8rot}), while typical newborn pulsars rotate with periods $P_{i} \gtrsim 10-100 \, {\rm ms}$ \citep{faucher:06,igoshev:13,gullon:14} much slower than breakup and too slow to affect  supernovae. Initial LIGO results also favor slow BH birth spins (Figure \ref{fig:BHspin}), as all but one BH merger is consistent with the aligned component of the BH spins $\chi_{\rm eff} = 0$.


The slow rotation rates of compact objects indicates a powerful source of AM transport in stars, because conservation of core AM from the main sequence invariably predicts compact object rotation rates near breakup. This is also consistent with the slower-than-expected asteroseismic rotation rates (\ref{fig:MRI1p8rot}) of intermediate-mass red giants that could be explained by AM transport due to baroclinic instabilities. {\bf We will therefore investigate the impact of baroclinic instabilities on the rotation rates of neutron stars and black holes originating from high-mass stellar progenitors, and determine whether the resulting AM transport can explain measurements from LIGO and young pulsars.} Previous stellar evolution predictions (e.g., \citealt{heger:00}) of compact object rotation rates have been based on prescriptions that incorrectly predict core rotation rates for intermediate-mass stars \citep{cantiello:14}. Upon calibration of the baroclinic AM transport using asteroseismic data, we will implement baroclinic torques into MESA models of massive stars, allowing for the first reliable predictions of compact object spin rates. By running a suite of massive star models, we will predict compact object spin rates as a function of progenitor mass, rotation rate, metallicity, etc. While their are many uncertainties in massive star evolution, the relative insensitivity of core physics to these uncertainties suggests that fairly accurate predictions are possible.

\subsubsection{Stochastic Spin-up}

The spin rate of compact objects can also be affected by waves inside their progenitors. In particular, stellar cores can be {\it spun-up} by the convectively excited waves, which carry huge fluxes of AM durig \cite{fullerwave:14} show that 

