%{\color{red} \subsubsection{White Dwarfs}}

Roughly 10 \% of WDs show strong surface magnetic fields ($B > 1 \, {\rm MG}$).
%The fraction of WDs with strongly magnetized surfaces is debated, but usually quoted to be of the order of 10\%.
Interestingly, magnetized WDs are systematically more massive \cite{Ferrario_2015}, which could be explained if the observed magnetic fields descend from a main sequence convective core dynamo \cite{Cantiello_2016}. This is because convective cores during the main sequence extend to mass coordinates above 0.6$M_\odot$ only in stars with initial mass $M 3M_\odot$, so that the entire mass of the WD descendant lies within the mass coordinate occupied by the MS convective core. Ohmic diffusion timescales are larger than the nuclear burning timescales and the WD cooling timescale, so magnetic fields are virtually confined in the mass coordinate where they have been produced \cite{Cantiello_2016}. In this scenario, stars with $M>3M_\odot$ produce WDs with $M_{\rm WD}>0.7M_\odot$ with strong surface magnetic fields, similar to the typical masses of observed magnetic WDs. Moreover, strong magnetic fields may exist within the interior of many WDs, even if not visible at their surface. Magnetic fields in excess of $10^4-10^5 G$ might suppress or modify typical WD g-mode pulsations. Observing the absence of pulsations in WDs observed by K2 and TESS that are located in regions predicted to be unstable, will help identifying and characterized internal magnetism. Together with existing observations of WD surface magnetic fields, this will \textbf{provide quantitative information about magnetic fields in the last phases of stellar evolution, allowing to reconstruct the full history of stellar magnetism}.