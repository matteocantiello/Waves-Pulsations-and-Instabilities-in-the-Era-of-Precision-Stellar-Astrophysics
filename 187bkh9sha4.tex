The \citet{Townsend:2017} study reveals a key missing ingredient in the main-sequence evolution of massive stars: \emph{while we have traditionally regarded pulsation as phenomenon that can shed light on the internal structure of these stars, we have overlooked the possibility that pulsation (specifically, heat-driven g modes) may actively be changing this structure}. We propose to address this through a comprehensive series of activities that establish, for the first time, the overall impact of heat-driven modes on massive star evolution. A first step, requiring little in the way of preparatory activities, will be to extend the \citet{Townsend:2017} simulations to the SPBSg stars (the blue region with $M \gtrsim 9\,M_{\odot}$ in Figure~\ref{f:xxx}); this will provide a preliminary assessment of whether g modes can be a significant transporter of angular momentum in B supergiants.

The next steps, however, will require a combination of computational and theoretical efforts, taking place under the guidance of co-PI Townsend with the assistance of graduate student Lopez. The angular momentum transport simulations by \citet{Townsend:2017} do not self-consistently account for the fact that the developing differential rotation profile of the model star will alter the properties (excitation, damping, etc) of the g mode responsible the transport, primarily via the action of the Coriolis force. The GYRE code already has the ability to model this feedback effect, but only approximately through the so-called traditional approximation of rotation (TAR) where some components of the Coriolis force are neglected. The advantage of the TAR is that the governing pulsation equations remain separable in all three coordinates, and therefore can be solved using similar numerical techniques to the non-rotating case \citep[see, e.g.,][for a demonstration of the TAR applied to rotating SPB stars]{Townsend:2005}. However, it neglects possible couplings between modes of different harmonic degree $\ell$, and moreover is only formally applicable to cases of uniform or near-uniform rotiation---certainly not the case shown, e.g., in Figure~YY!

To move beyond the TAR, we propose to extend the GYRE code to implement the Coriolis-force effects without recourse to any approximations. Mathematically, we will follow the spectral expansion approach outlined e.g. by \citep{Lee:2001}, which represents the angular dependence of pulsation variables at each radius via a series expansion in spherical harmonics. We have already completed a proof-of-concept exercise to confirm that this approach can readily be implemented with GYRE's existing architecture (at its heart, an application-agnostic boundary eigenvalue problem solver), and therefore we are confident that the necessary code development can be completed within the first year of the project. The one area that may require some further theoretical innovation is in mode identification: with spectral approaches, modes with different values of $\ell$ appear in the same eigenfrequency space, and a robust procedure will have to be devised to disentangle the dipole, quadrupole, etc modes from each other.

With this code development completed, specific areas we plan to address are as follows:
\begin{itemize}
\item how does the rotation change the excitation of the g mode driving the rotation? Do g modes self-
\item 