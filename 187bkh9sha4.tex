The next steps, however, will require a combination of computational and theoretical efforts, taking place under the guidance of co-PI Townsend with the assistance of graduate student Lopez. The angular momentum transport simulations by \citet{Townsend:2017} do not self-consistently account for the fact that the developing differential rotation profile of the model star will alter the properties (excitation, damping, etc) of the g mode responsible the transport. The GYRE code already has the ability to model this feedback effect, but only approximately through the so-called traditional approximation of rotation (TAR) where some components of the Coriolis force are neglected. The advantage of the TAR is that the governing pulsation equations remain separable in all three coordinates, and therefore can be solved using similar numerical techniques to the non-rotating case \citep[see, e.g.,][for a demonstration of the TAR applied to rotating SPB stars]{Townsend:2005}. However, it neglects possible couplings between modes of different harmonic degree $\ell$, and moreover is only formally applicable to cases of uniform or near-uniform rotiation