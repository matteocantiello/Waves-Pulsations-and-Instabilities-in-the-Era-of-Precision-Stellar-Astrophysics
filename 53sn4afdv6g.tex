Thanks to a new asteroseismic technique developed by our team, we are now in the position to detect strong {\it internal} magnetic fields. This technique was originally developed for red giants stars by utilizing observations of mixed modes, which are oscillations that behave as pressure waves (p-modes) near the stellar surface and gravity waves (g-modes) in the stellar core. \citep[e.g][]{Beck_2011}. The presence of a strong magnetic field in the core is able to alter the propagation of the gravity waves (in particular the dipolar $\ell =1$ modes),
leading to mode energy loss in the core which decreases the pulsation amplitude at the surface \citep{Fuller_2015}. This has also been demonstrated by numerical simulations of the interaction between internal gravity waves and magnetic fields using the Dedalus code \citep{Lecoanet_2016}. Red giants with strong internal magnetic fields ($B \! \gtrsim \! 10^5 \, {\rm G}$) can thus be identified by the presence of depressed oscillation modes in their power spectra \citep{Fuller_2015,Stello_2016}, 
%Recent theoretical work from \citet{2017MNRAS.467.3212L} also confirms that  strong magnetic fields can provide an efficient mean of damping stellar oscillations,
although it is still possible that other unknown physical processes are at work in some red giants with depressed dipole modes \citet{Mosser_2017}. When applied to more than 3000 red giants observed by the Kepler satellite, this technique revealed that strong magnetic fields are present in the cores of roughly 50\% of RGB stars with $M \! > \! 1.5 \, M_\odot$, but are very rare in stars with M $< 1.1M_\odot$ \citep{Stello_2016,Stello_2016a}. This is interpreted as the effect of MS core-dynamo-generated magnetic fields in the convective cores of $M \! > \! 1.1 \, M_\odot$ stars, suggesting that a large fraction of main sequence OBA stars could host strong internal magnetic fields.
