This technique was developed for red giants stars first, and utilizes the observations of mixed modes. Mixed modes are
oscillations that behave as pressure waves (p-modes) near the stellar surface and gravity waves (g-modes) in the stellar core.
The study of these oscillations allows to determine properties of both core and stellar envelope (e.g. Beck et al. 2012).
The presence of a strong magnetic field in the core is able to alter the propagation of the gravity waves (in particular the dipolar $\ell =1$ modes),
trapping mode energy in the core and effectively decreasing its visibility \citep{Fuller_2015}.
For some simple field configurations, this theoretical result is supported  by numerical simulations of the interaction between
internal gravity waves and magnetic fields using the Dedalus code \citep{Lecoanet_2016} (Lecoanet et al. 2017).
Red giants with strong internal magnetic fields (B $> 10^5$ G) can thus be identified by the presence of suppressed oscillation modes in their oscillation spectra.
