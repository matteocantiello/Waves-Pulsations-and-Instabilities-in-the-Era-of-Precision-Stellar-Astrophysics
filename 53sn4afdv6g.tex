Thanks to a new asteroseismic technique developed by our team, we are now in the position to also detect strong internal magnetic fields.
This technique was developed for red giants stars first, and utilizes the observations of mixed modes. Mixed modes are
oscillations that behave as pressure waves (p-modes) near the stellar surface and gravity waves (g-modes) in the stellar core.
The study of these oscillations allows to determine properties of both core and stellar envelope \citep[e.g][]{Beck_2011}.
The presence of a strong magnetic field in the core is able to alter the propagation of the gravity waves (in particular the dipolar $\ell =1$ modes),
leading to mode energy loss in the core which decreases its visibility at the surface \citep{Fuller_2015}.
For some simple field configurations, this theoretical result is supported  by numerical simulations of the interaction between
internal gravity waves and magnetic fields using the Dedalus code \citep{Lecoanet_2016}.
Red giants with strong internal magnetic fields (B $> 10^5$ G) can thus be identified by the presence of depressed oscillation modes in their oscillation spectra \citep{Fuller_2015,Stello_2016}. Recent work from \citet{2017MNRAS.467.3212L} suggest that indeed magnetic fields can provide an efficient means of damping stellar oscillations, although it is still  possible that some other physical processes could result in depressed dipole modes in a subset of depressed dipole modes \citet{Mosser_2017}.
