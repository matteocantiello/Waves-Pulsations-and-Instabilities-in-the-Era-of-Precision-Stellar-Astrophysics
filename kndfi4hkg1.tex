\section{The Era of Precision Stellar Astrophysics}

Stellar astrophysics is undergoing a renaissance driven by new observational and theoretical capabilities. Large-scale time-domain surveys have uncovered new classes of stellar explosions that provide insights into how stars evolve and end their lives.  The goal of precisely measuring planet properties using radial velocity and transit methods has led to the need for more precise measurements of stellar properties.  Efforts to reconstruct the assembly history of the Milky Way (galactic archeology) using large-scale spectroscopic surveys (e.g., APOGEE) requires precise measurements of stellar masses, abundances, and distances.   The detection of merging black holes and neutron stars by LIGO has provided compelling new constraints on black hole spin and binary stellar evolution. 

Arguably, however, the most dramatic advance in understanding stellar structure and evolution in the last decade has been advances in asteroseismology, i.e., the study of waves in stars.   The reason is that asteroseismology provides stringent constraints on the {\em interior} structure of large samples of stars, in contrast to traditional methods that are restricted to global (mass, radius) and/or surface (abundances) properties.   Over the past NN years, the exquisite photometry produced by the Kepler satellite revolutionized asteroseismology.  It enabled entirely new constraints on stellar structure, internal differential rotation, internal magnetism, and chemical mixing in stars (REFS).  It also launched the largely new field of 'tidal asteroseimology' through the detection of pulsations excited by the time varying gravity in close, eccentric binary systems (REFS). 

Waves in stars play two distinct roles;  In addition to providing a probe of stellar interiors, waves  can also dramatically affect the structure and evolution of the star itself.   Waves can efficiently transporting angular momentum and energy in stellar interiors and produce chemical mixing on top of that produced by convection or other hydrodynamic or magnetohydrodynamic instabilities.   Indeed, wave transport of angular momentum may be critical for understanding the rotation profile of a wide range of stars, from the sun to neutron stars and black holes (REF).   Wave transport of energy may in turn be a key for understanding mass loss in a wide range of stars (e.g., the sun, red supergiants, and pre core-collapse massive stars).   

Our proposed network is focused on 1.  understanding how asteroseismology can be used as a probe of stellar interiors and interpreting and modeling current and forthcoming asteroseismology results.  By understanding the physics needed to explain asteroseismology observations, the improvements in our understanding of stars enabled by asteroseismology impacts a wide range of other astrophysics, including exoplanets, galactic structure and compact objects.   


2.  understanding the role of waves in producing angular momentum transport and chemical mixing in stellar interiors.    

In turn, the improvements in our understanding of stars enabled by asteroseismology impacts a wide range of other astrophysics, including exoplanets, galactic structure and compact objects.   


The team associated with our proposed network has been at the forefront of the effort to interpret Kepler results, their implications for stellar structure and evolution, and their implications for properties of compact objects produced by stellar evolution (REFS).   
