\section{The Era of Precision Stellar Astrophysics}
Broader intro

Stellar astrophysics is undergoing a renaissance driven by new observational and theoretical capabilities. Large-scale time-domain surveys have uncovered new classes of stellar explosions that provide insights into how stars evolve and end their lives.  Precise measurements of planet properties using radial velocity and transit methods require precise measurements of the properties of the host star.  Likewise, efforts to reconstruct the assembly history of the Milky Way (galactic archeology) require precise measurements of stellar masses, abundances, radii, and distances.    The most dramatic advance in understanding stellar structure and evolution in the last decade has arguably been produced by advances in asteroseismology, which  provide stringent constraints on the interior structure and evolution of stars.    in particular, 

The exquisite photometry produced by the Kepler satellite revolutionized asteroseismology.  

As we will discuss throughout this proposal, Kepler enabled entirely new constraints on stellar structure, internal differential rotation, internal magnetism, and chemical mixing in stars (REFS).  It also launched the largely new field of 'tidal asteroseimology' through the detection of pulsations excited by the time varying gravity in close, eccentric binary systems (REFS). The team associated with our proposed network has been at the forefront of the effort to interpret these results, their implications for stellar structure and evolution, and their implications for properties of compact objects produced by stellar evolution (REFS).   

