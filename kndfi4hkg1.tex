\section{The Era of Precision Stellar Astrophysics}

Stellar astrophysics is undergoing a renaissance driven by new observational and theoretical capabilities. Large-scale time-domain surveys have uncovered new classes of stellar explosions that provide insights into how stars evolve and end their lives.  Precise measurements of planet properties using radial velocity and transit methods require correspondingly precise measurements of stellar properties.  Efforts to reconstruct the assembly history of the Milky Way (galactic archeology) using large-scale spectroscopic surveys (e.g., APOGEE) require precise measurements of stellar masses, abundances, and distances.   And the detection of merging black holes and neutron stars by LIGO has provided compelling new constraints on black hole spin and binary stellar evolution.   

On the theoretical side, the open-source stellar evolution code MESA and pulsation code GYRE developed by members of this proposal have become indispensible tools in the field, and enable  [not quite sure what i want to say here in in terms of mesa/gyre ...]   Moreover, it is increasingly possible to study aspects of the three-dimensional structure of stars using targeted numerical simulations (e.g., convection, stellar atmospheres, and rotationally-driven instabilities).  These targeted studies can then be used to develop more accurate models of this physics in one-dimensional stellar evolution codes.

Arguably, however, the most dramatic advance in understanding stellar structure and evolution in the last decade has been advances in asteroseismology, i.e., the study of waves in stars.   The reason is that asteroseismology provides stringent constraints on the {\em interior} structure of large samples of stars, in contrast to traditional methods that are restricted to global (mass, radius) and/or surface (abundances) properties.   Over the past $\sim 8$ years, the exquisite photometry produced by the Kepler satellite revolutionized asteroseismology.  The bread and butter of asteroseismology is precise stellar masses and radii (and hence, indirectly, ages and distances), which Kepler provided for ?? red giants.   Perhaps more remarkably, however, Kepler also diagnosed internal differential rotation, internal magnetism, and enhanced chemical mixing in a large number of stars (REFS).  Kepler also launched the largely new field of 'tidal asteroseimology' through the detection of pulsations excited by the time varying gravity in close, eccentric binary systems (REFS). 

In addition to providing a {\em probe} of stellar interiors, waves in stars can also dramatically {\em modify} the structure and evolution of the star itself.   Waves can efficiently transport angular momentum and energy in stellar interiors and produce chemical mixing on top of that produced by convection or other instabilities.   Indeed, wave transport of angular momentum may be critical for understanding the rotation profile of a wide range of stars, from the sun to neutron stars and black holes (REF).   Wave transport of energy may in turn be a key for understanding mass loss in a wide range of stars (e.g., the sun, red supergiants, and pre core-collapse massive stars).   

Our proposed network is focused on 1.  understanding novel ways that asteroseismology can be used as a probe of stellar interiors, 2. interpreting and modeling current and forthcoming asteroseismology results and their implications for other areas of astrophysics, and 3.  understanding the role that waves play in modifying stellar structure and evolution.  The team associated with our proposed network has been at the forefront of the effort to understand the implications of Kepler asteroseismology results and the physics of waves in stars more generally (REFS).  Subsets of our group have successfully collaborated before so we are confident in our ability to carry out the interconnected research program required for a successful TCAN effort.   

We believe that the time is particularly ripe for a focused effort in this area.  Kepler observations have been synthesized and raise a number of interconnected puzzles that impact many other areas of astrophysics.  For example, how can we consistently connect the internal rotation and internal magnetic fields of stars measured by asteroseismology to the rotation of white dwarfs, pulsars, and black holes?  Moreover, the coming 5 years will see new dramatic progress in this area. In particular, Gaia distances will remove one of the largest uncertainties associated with studying stars in the Milky Way.  The TESS satellite will expand the range of asteroseismology from low mass stars (the primary targets in the Kepler fields) to massive stars in a range of evolutionary states.   And JWST will provide unparalleled photometry and spectroscopy of cool stars.    We believe that the improvements in our understanding of stars enabled by our network will be interesting and important in and of themselves.  Equally importantly, however, this proposed work will impact a wide range of other astrophysics, including studies of exoplanets, galactic structure and compact objects.   


