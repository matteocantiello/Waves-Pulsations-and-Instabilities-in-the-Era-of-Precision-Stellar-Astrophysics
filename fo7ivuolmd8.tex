{\color{purple}
\subsection{Convective Wave Excitation}
}

We do not have a satisfactory theory for the excitation of gravity waves by convection. This is a key uncertainty in estimating the angular momentum and chemical mixing due to these waves. We want to determine the wave flux spectrum, which is expected to take the form of a power law,
\begin{equation}
\frac{d F_{\rm w}}{d\log \omega \log k_h} = M_{\rm c}F_{\rm c} \left(k_h H_{\rm c}\right)^{a} \left(\frac{\omega}{\omega_{\rm c}}\right)^{-b}.
\end{equation}
There is broad agreement that the total wave flux is roughly equal to the convective Mach number $M_{\rm c}$ times the convective flux $F_{\rm c}$ (**cite**). However, many theories have been proposed for the dependence of the wave flux on the frequency $\omega$ and horizontal wavenumber $k_h$ of the waves (**cite**). These theories predict a very strong dependence on the frequency, e.g., $b\sim 7$. Such a spectrum has never been observed in simulations, which find more shallow spectra with $b\sim 3$ (**cite Tami**). The dependence on the frequency is especially important as low frequency waves (which carry most of the wave flux) are often strongly damped, and do not influence stellar evolution. 

** Do we want to include this?**

A major obstacle in determining this spectrum is different codes find drastically different wave spectra for simulations that appear to have identical parameters (Tami Rogers, private communications). \textbf{We will develop a convective wave generation benchmark problem, and ensure all community codes arrive at a validated, unique solution.} Using Dedalus, we will find a two dimensional problem which can be easily studied with a variety of numerical methods, e.g., spectral (Dedalus), Godunov (Athena), high-order finite difference (Pencil), fully-implicit (MUSIC), etc. We will use our experience in developing a hydrodynamics benchmark problem (Lecoanet et al 2015) to find a well-posed problem that can be accurately solved with many different codes.

Once we have confirmed that waves found in simulations are physical, rather than numerical, we will run a series of three-dimensional simulations specifically designed to test theories of wave generation. The theories assume a separation of scales between the frequency of large-scale convective eddies ($\omega_{\rm c}$) and the Brunt-Vaisala frequency in the radiative zone ($N$). This also requires high spatial resolution because the turbulence cascade must be resolved between these two timescales. Preliminary tests suggest we can use Dedalus to run simulations with $N/\omega_{\rm c}\sim 30$, which should be enough to accurately measure the wave spectrum. We will also study the effects of rotation on wave generation. This is important because rotation will change both the properties of the waves and the convection.

\color{blue}
\subsection{Directly Observing Convectively Excited Internal Gravity Waves}

There is currently no unambiguous detection of  internal gravity waves stochastically excited by convection in any type of star.  Observed internal gravity waves  are excited by other means, e.g., heat driven instabilities in SPB stars. Nonetheless, internal gravity waves excited by convection are believed to be a key source of angular momentum and energy transport, and chemical mixing, in stellar interiors.   Such waves have been invoked to explain a wide range of stellar observations, including the rotation of the solar interior \citep{kumar1999}, the rotation of pulsars \citep{fuller2015}, enhanced chemical mixing inferred from asteroseismology (REF), and large pre-supernova mass loss inferred from circumstellar interaction in core-collapse supernovae \citep{qs2012}.   Directly calibrating theoretical models of internal gravity wave excitation would thus impact a particularly wide range of stellar science.

Calculations of the observational manifestation of convectively excited internal gravity waves in main sequence stars suggest that they are most readily detectable in massive stars where core convection launches waves towards the surface, producing potentially detectable luminosity variations and smaller, more difficult to detect, velocity variations \citep{samadi2010,shiode2013}.   These calculations are uncertain, however, because of the uncertain spectrum of waves excited by convection (see \S ??).

There are some indications in the photometry of massive stars with COROT for excess variability consistent with that expected from internal gravity waves \citep{aerts2015}.   TESS will dramatically expand the range of massive stars with high precision photometry, and create remarkable new opportunities for directly observing the surface manifestation of internal gravity waves.   
Our proposed network is ideally positioned to interpret these new observational results and assess their implications.   Moreover, we will utilize the numerical simulations of wave excitation described in \S ?? to re-assess with state-of-the art models the observational manifestation of convectively excited waves in both main sequence and evolved stars:   what are the luminosity and velocity perturbations induced by waves at the stellar surface and how do the predictions compare to observations?   Which phases of stellar evolution and which mass stars are the most propitious for observing stochastically excited internal gravity waves?  These calculations will use Dedalus to determine the spectrum of excited waves,waves, MESA to calculate the stellar models, MESA to calculate the stellar models Gyre to determine the wave damping rates and surface luminosity and velocity perturbations.  They thus require the interaction of all of the nodes of our proposed network.

could add more?   Iron bump excitation
