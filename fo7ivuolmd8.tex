{\color{purple}
\subsection{Convective Wave Excitation}
}

We do not have a satisfactory theory for the excitation of gravity waves by convection. This is a key uncertainty in estimating the angular momentum and chemical mixing due to these waves. We want to determine the wave flux spectrum, which is expected to take the form of a power law,
\begin{equation}
\frac{d F_{\rm w}}{d\log \omega \log k_h} = M_{\rm c}F_{\rm c} \left(k_h H_{\rm c}\right)^{a} \left(\frac{\omega}{\omega_{\rm c}}\right)^{-b}.
\end{equation}
There is broad agreement that the total wave flux is roughly equal to the convective Mach number $M_{\rm c}$ times the convective flux $F_{\rm c}$ (**cite**). However, many theories have been proposed for the dependence of the wave flux on the frequency $\omega$ and horizontal wavenumber $k_h$ of the waves (**cite**). These theories predict a very strong dependence on the frequency, e.g., $b\sim 7$. Such a spectrum has never been observed in simulations, which find more shallow spectrum with $b\sim 3$ (**cite Tami**). The dependence on the frequency is especially important as low frequency waves (which carry most of the wave flux) are often strongly damped, and do not influence stellar evolution. 

** Do we want to include this?**

A major obstacle in determining this spectrum is different codes find drastically different wave spectra for simulations that appear to have identical parameters (Tami Rogers, private communications). \textbf{We will develop a convectively wave benchmark problem }
