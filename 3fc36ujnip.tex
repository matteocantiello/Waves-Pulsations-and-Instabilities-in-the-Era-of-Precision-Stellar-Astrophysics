When applied to more than 3000 red giants observed by the Kepler satellite, this technique revealed that strong magnetic fields (B $> 10^5$ G) are present in the core of roughly 50\% of RGB stars with $M \! > \! 1.5 \, M_\odot$ \citep{Stello_2016,Stello_2016a}. Conversely, strong internal magnetic fields are very rare in stars with M $< 1.1M_\odot$. This is interpreted as the effect of MS core-dynamo-generated magnetic fields in the convective cores of $M \! > \! 1.1 \, M_\odot$ stars, and it shows that a potentially very large fraction of main sequence OBA stars could host strong internal magnetic fields.When applied to more than 3000 red giants observed by the Kepler satellite, this technique revealed that strong magnetic fields (B $> 10^5$ G) are present in the core of roughly 50\% of RGB stars with $M \! > \! 1.5 \, M_\odot$ \citep{Stello_2016,Stello_2016a}. Conversely, strong internal magnetic fields are very rare in stars with M $< 1.1M_\odot$. This is interpreted as the effect of MS core-dynamo-generated magnetic fields in the convective cores of $M \! > \! 1.1 \, M_\odot$ stars, and it shows that a potentially very large fraction of main sequence OBA stars could host strong internal magnetic fields.



