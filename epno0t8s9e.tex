\section{Computation and Codes}
    
    
{\color{green}    
\subsection{MESA}}
% MC: From SPIDER TCAN
\textbf{MESA} Modules for Experiments in Stellar
Astrophysics, solves the 1D fully coupled structure and composition
equations governing stellar evolution \cite{Paxton_2010} \cite{Paxton_2013} \cite{Paxton_2015} \citep{paxton11,Paxton:2013}. It
is based on an implicit finite difference scheme with adaptive mesh
refinement and sophisticated timestep controls; 
state-of-the-art modules provide equation of state, opacity, nuclear
reaction rates, and element diffusion.
%, as well as atmosphere boundary conditions, and
%an innovative algorithm for mass loss and mass accretion. 
It is 
available under an open-source license and has  attracted $>800$ 
 users.
%, witnessed over 5,000 downloads
%from \url{http://mesa.sourceforge.net/}, started an annual Summer
%School program \url{http://cococubed.asu.edu/mesa_summer_school_2013},
%and provided a portal for the community to openly share knowledge
%\url{http://mesastar.org}.

%%MESA employs contemporary
%%numerical approaches and is written with present and future
%%shared-memory, multi-core, and multi-thread architectures in mind.
%%MESA combines the numerical and physics modules for simulations of a
%%wide range of stellar evolution scenarios ranging from very-low mass
%% to massive stars, including advanced evolutionary phases leading up to
%% explosions.  It solves the fully coupled structure and composition
%% equations simultaneously.  It uses adaptive mesh refinement and
%% sophisticated timestep controls, and supports shared memory
%% parallelism based on OpenMP.  State-of-the-art modules provide
%% equation of state, opacity, nuclear reaction rates, and element
%% diffusion, as well as atmosphere boundary conditions, and an
%% innovative algorithm for mass loss and mass accretion.  MESA has
%% attracted over 500 registered users, witnessed over 5,000 downloads
%% from \url{http://mesa.sourceforge.net/}, started an annual Summer
%% School program \url{http://cococubed.asu.edu/mesa_summer_school_2013},
%% and provided a portal for the community to openly share knowledge
%% \url{http://mesastar.org}.q  MESA's domain of applicability continues
%% to grow, just recently extended to giant planets, oscillations, and
%% rotation \citep{paxton_2013_aa}.



   
    
    
{\color{brown}    
\subsection{GYRE}}
  
\textbf{GYRE} solves the 3D linearized hydrodynamic
equations to calculate the nonradial, nonadiabatic oscillation
spectrum of differentially rotating stars \citep{bard11}. It is based
on a spherical harmonic method in angular directions coupled with a
sixth-order Magnus scheme in the radial direction. Initial tests on
$\gtrsim 100$ processor architectures reveal near-linear parallel
scaling. It is available under open-source license, and is the only public oscillation code capable of
considering the effects of nonadiabaticity and/or differential
rotation \citep{Townsend_2013}.

    
