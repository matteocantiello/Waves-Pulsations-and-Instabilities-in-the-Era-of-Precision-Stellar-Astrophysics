\section{Computation and Codes}
    
    
{\color{green}    
\subsection{MESA}
}

\textbf{MESA}, Modules for Experiments in Stellar
Astrophysics, solves the 1D fully coupled structure and composition
equations governing stellar evolution \cite{Paxton2010,Paxton2013,Paxton2015}. It
is based on an implicit finite difference scheme with adaptive mesh
refinement and sophisticated timestep controls; 
state-of-the-art modules provide equation of state, opacity, nuclear
reaction rates, and element diffusion.
%, as well as atmosphere boundary conditions, and
%an innovative algorithm for mass loss and mass accretion. 
It is 
available under an open-source license and has  attracted $>800$ 
 users.

     
    
{\color{brown}    
\subsection{GYRE}}
 \textbf{GYRE} solves the 3D linearized hydrodynamic
equations to calculate the nonradial, nonadiabatic oscillation
spectrum of differentially rotating stars \cite{2013MNRAS.435.3406T}. It is based
on a spherical harmonic method in angular directions coupled with a
sixth-order Magnus scheme in the radial direction. Initial tests on
$\gtrsim 100$ processor architectures reveal near-linear parallel
scaling. It is available under open-source license, and the only public oscillation code capable of
considering the effects of nonadiabaticity and/or differential
rotation.
