\section{Computation and Codes}
    
\subsection{MESA}

%\textbf{MESA}, Modules for Experiments in Stellar
%Astrophysics, solves the 1D fully coupled structure and composition
%equations governing stellar evolution \cite{Paxton2010,Paxton2013,Paxton2015}. It
%is based on an implicit finite difference scheme with adaptive mesh
%refinement and sophisticated timestep controls; 
%state-of-the-art modules provide equation of state, opacity, nuclear
%reaction rates, and element diffusion.
%It is 
%available under an open-source license and has  attracted $>800$ 
% users.

One of our key computational tools for this proposed effort will be
Modules for Experiments in Stellar Astrophysics (MESA, \href{http://mesa.sourceforge.net/}{http://mesa.sourceforge.net/}
), an open source code for stellar
structure and evolution. This code is a robust, efficient set of
thread-safe libraries used for a wide range of applications in
computational stellar astrophysics. It is now the most widely used
publicly available computational tool for stellar structure and
evolution, with over 800 registered users.

Though continued development is led by computer scientist Bill Paxton,
a Senior Fellow in Computational Astrophysics at KITP, there is now an
emergent set of other developers (including Cantiello and Townsend of
this proposal) who contribute to enhancing MESA's capabilities through
code development. An even larger group of scientists work to carefully
document the development progress in a series of ``instrument papers" \cite{Paxton2010,Paxton2013,Paxton2015,2017arXiv171008424P}
 that three of the investigators
(Bildsten, Cantiello and Townsend) have been deeply involved in.  All
of these efforts are in the open, with the source code immediately
available to the international astrophysical community, allowing for
continued engagement of new developers. MESA is also being used in
graduate courses across the country, allowing for a marked improvement
in the education of the next generation of astrophysicists.

In the most recent instrument paper \citep{2017arXiv171008424P} a number of
key new capabilities were introduced. The one that is most applicable
to our proposed work is the new approach to locating convective
boundaries that now yields reliable values of the convective core mass
during both hydrogen and helium burning phases. In addition, the
inclusion of an approximate Riemann solver now allows MESA to capture
shocks and conserve energy to high accuracy that allows for studies of
non-local transport within stars. Active MESA development relevant to
this proposal includes a more careful incorporation of low density and
low temperature equations of state to enable simulations of
non-linear radial stellar pulsations including Cepheid and red supergiant pulsations.
\  
    
\subsection{GYRE}

\textbf{GYRE} solves the 3D linearized hydrodynamic equations to calculate the nonradial, nonadiabatic oscillation spectrum of stars \cite{2013MNRAS.435.3406T,Townsend:2017aa}. The angular dependence of perturbations are represented via spherical harmonics, while the radial dependence is found by solving a boundary eigenvalue problem (BEVP) on a discrete grid using a multiple shooting scheme. A variety of one-step methods are implemented for the shooting (Magnus, collocation, mono-implicit Runge Kutta), with orders up to $6^{\rm th}$ for high accuracy. The large, sparse (staircase-form) linear system resulting from the discretization is solved using a structured factorization algorithm parallelized in OpenMP. GYRE handles the effects of (mild) differential rotation using the traditional approximation of rotation (TAR); optical thinness using the Eddington approximation; and can deal with specialized circumstances such as non-equilibrium nuclear burning, tidal forcing, and extremely high-order (short radial wavelength) oscillations. To achieve the goals of this proposal, we will continue to develop GYRE to include the full effects of rotation and convection on stellar pulsations.

GYRE was released under an open-source license in 2013, and since then has been adopted by over 50 groups worldwide as a general-purpose tool for asteroseismology. In modular form it is integrated into the MESA code, allowing on-the-fly optimization of stellar parameters (e.g., mass, composition, mixing length) to match sets of observed frequencies. 
