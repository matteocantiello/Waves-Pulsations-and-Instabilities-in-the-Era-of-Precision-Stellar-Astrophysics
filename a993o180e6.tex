 This work may also begin to better illuminate the connection between
 pulsations and large mass loss rates highlighted in the original work of \citet{Yoon_2010}. 
 They showed that the amplitudes of
 pulsations may become large enough to drive a time-dependent
 'super-wind' adequate to yield a substantial loss of the outer
 envelope of the RSG, substantially impacting how the resulting core
 collapse supernovae would appear to the observer. 