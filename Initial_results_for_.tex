Initial results for baroclinic AM transport are promising. Fuller et al. 2018 (in prep) have implemented the torque given by equation \ref{numri} into MESA to model the angular momentum evolution of intermediate-mass stars. Figure \ref{fig:MRI1p} shows the core rotation rate of these models compared to asteroseismic observations, demonstrating a close match from the main sequence all the way to the white dwarf stage. Baroclinic instabilites produce core rotation rates roughly 10 times slower than the next most important source of AM transport in these models, provided by the Taylor-Spruit dynamo. These models are relatively insensitive to initial spin rates or the value of $\alpha$ due to the strong dependence of $\nu_{\rm AM}$ on $\Omega_{\rm core}$, causing core rotation rates to naturally evolve into a narrow range in post-MS stars.

{\bf We plan to follow up on these encouraging preliminary results by performing a detailed investigation of baroclinic instabilities in stars, and their effect on mixing of AM and chemical species.}