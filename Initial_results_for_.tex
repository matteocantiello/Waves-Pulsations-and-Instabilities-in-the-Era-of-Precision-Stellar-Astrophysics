
Initial results for baroclinic AM transport are promising. Fuller et al. 2018 (in prep) have implemented the torque given by equation \ref{numri} into MESA to model the angular momentum evolution of intermediate-mass stars. Figure \ref{fig:MRI1p8rot} shows the core rotation rate of these models compared to asteroseismic observations, demonstrating a close match from the main sequence all the way to the white dwarf stage. Baroclinic instabilites produce core rotation rates roughly 10 times slower than the next most important source of AM transport in these models, provided by the Taylor-Spruit dynamo. These models are relatively insensitive to initial spin rates or the value of $\alpha$ due to the strong dependence of $\nu_{\rm AM}$ on $\Omega}$, causing core rotation rates to naturally evolve into a narrow range in post-MS stars.

{\bf We plan to follow up on these encouraging preliminary results by performing a detailed investigation of baroclinic instabilities in stars, and their effect on mixing of AM and chemical species.} The first step is to perform numerical simulations of baroclinic instabilities using Dedalus. These simulations are non-trivial, because baroclinic instabilities necessarily cause latitudinal and non-axisymmetric flows, and therefore three dimensional geometry is required. The instabilities also occur on very short length scales in real stars, hence the simulations must be carefully constructed to capture the instability and preserve the correct separation of scales. The first goal of the simulations is to check that growth rates and lengthscales match analytical calculations. The second goal is to capture the nature of the turbulent saturated state of the instability, including the turbulent kinetic energy, eddy sizes, and turnover times. Finally we will calibrate the angular momentum and chemical mixing induced by baroclinic instabilities, numerically measuring the appropriate $\alpha$ for use in equation \ref{numri}.


