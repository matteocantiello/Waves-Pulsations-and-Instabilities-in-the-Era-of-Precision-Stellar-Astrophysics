{\color{blue}
\subsection{A New Era of Asteroseismology}

The exquisite photometry produced by the Kepler satellite revolutionized asteroseismology.   As we will discuss throughout this proposal, Kepler enabled entirely new constraints on stellar structure, internal differential rotation, internal magnetism, and chemical mixing in stars (REFS).  It also launched the largely new field of 'tidal asteroseimology' through the detection of pulsations excited by the time varying gravity in close, eccentric binary systems (REFS). The team associated with our proposed network has been at the forefront of the effort to interpret these results, their implications for stellar structure and evolution, and their implications for properties of compact objects produced by stellar evolution (REFS).   

One limitation of the original Kepler mission from the perspective of asteroseismology is that the modest sky coverage led to Kepler observing primarily low mass stars, including low mass red giants and slowly pulsating B-stars (SPBs).  K2 and COROT observations have demonstrated the exciting potential of asteroseismology in more massive stars, but in just a few stars.   In the coming few years, however, the Transiting Exoplanet Survey Satellite (TESS) will enable an entirely new revolution in asteroseismology through its all sky survey of bright stars, including a large sample of stars with multi-year photometry in the polar continuous viewing zone.   This will lead to new asteroseismology results for wide range of previously unexplored stellar masses and stages of stellar evolution.   In particular, TESS will observed 1.  Massive main sequence stars 2. More massive red giants/sub giants 3. 3.  new classes of tidally excited pulsators, particularly more massive stars, 4. subgiant stars.   The first three classes were absent in the Kepler fields, while the last will be helped by the shorter TESS cadence.   Our proposed theoretical network is positioned to lead the interpretation of the forthcoming TESS asteroseismology results.  

In addition to detailed asteroseismology of some targets, the combination of TESS and GAIA will dramatically our knowledge of where stars in the HR diagram pulsate.   This zeroth order information is actually a strong constraint on stellar structure and evolution, in particular for more evolved parts of the HR diagram where whether the star pulsates depends on its previous evolutionary 

tess -- lots of info about instabilty strips from TESS but not nec. detailed asteroseismology.  what science can we do with instablity strips, especially in more evolved pars of HR disgram  probes of stellar structure and evolution.   need to understand some physics [convection-pulsation] to extract this information.   

work some of RSGs, etc. here




}