{\color{blue}
\subsection{New Opportunities in Asteroseismology}}

One limitation of the original Kepler mission from the perspective of asteroseismology is that the modest sky coverage led to Kepler observing primarily low mass stars, including low mass red giants and slowly pulsating B-stars (SPBs).  K2 and COROT observations have demonstrated the exciting potential of asteroseismology in more massive stars, but in just a few stars.   In the coming few years, however, the Transiting Exoplanet Survey Satellite (TESS) will enable an entirely new revolution in asteroseismology through its all sky survey of bright stars, including a large sample of stars with multi-year photometry in the polar continuous viewing zone.   This will lead to new asteroseismology results for wide range of previously unexplored stellar masses and stages of stellar evolution.   In particular, TESS will observe 1.  Massive main sequence stars 2. More massive red giants/sub giants 3. 3.  new classes of tidally excited pulsators, particularly more massive stars, 4. subgiant stars.   The first three classes were absent in the Kepler fields, while the last will be helped by the shorter TESS cadence.   Our proposed theoretical network is positioned to lead the interpretation of the forthcoming TESS asteroseismology results.  

In addition to detailed asteroseismology of some targets, the combination of TESS and GAIA will dramatically improve our knowledge of where stars in the HR diagram pulsate ('instability strips').   This zeroth order information is  a strong constraint on stellar structure and evolution, in particular for more evolved parts of the HR diagram where pulsational instability depends a star's previous evolutionary history. Extracting the full science out of the forthcoming instability strip results requires an improved understanding of both the theory of stellar pulsations (e.g., excitation vs. damping rates) and stellar structure and evolution -- this in turn requires our network's combined expertise in these areas (e.g., MESA and Gyre).
