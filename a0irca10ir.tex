{\color{blue}
\subsection{A New Era of Asteroseismology}

The exquisite photometry produced by the Kepler satellite (including the K2 mission) has revolutionized asteroseismology.   As we will discuss throughout this proposal, Kepler enabled entirely new constraints on stellar structure, internal differential rotation, internal magnetism, chemical mixing. (REFS).  It also launched the largely new field of 'tidal asteroseimology' through the detection of stellar pulsations excited by the time varying gravity in eccentric boinary   The team associated with our proposed network has been at the forefront of the effort to interpret these results, their implications for stellar structure and evolution, and the properties of compact objects produced by stellar evolution (REFS).   

One limitation of the original Kepler 

tess -- lots of info about instabilty strips from TESS but not nec. detailed asteroseismology.  what science can we do with instablity strips, especially in more evolved pars of HR disgram  probes of stellar structure and evolution.   need to understand some physics [convection-pulsation] to extract this information.   Tess will observe: 1. Massive main sequence stars 2. More massive red giants/sub giants 3. Subgiant stars 4.  new classes of tidally excited pulsators, paritcularly more massive stars.   The first two were absent in the Kepler field, while the last will be helped by the shorter TESS cadence.

work some of RSGs, etc. here




}