\section{Project Schedule and Management Plan}

The institutional PIs will be assisted by a graduate student to achieve our science goals, except for the Princeton node where we will hire a postdoc to run and analyze the more challenging Dedalus simulations. To achieve the data comparisons mentioned below, we will continue established collaborations with asteroseismic observers Conny Aerts, Dennis Stello, Gerald Handler, and JJ Hermes. We hope to achieve our research goals as follows:

Year 1: Make predictions for locations of instability strips for comparison with GAIA/TESS data, and predict fractions of non-pulsating stars due to magnetic suppression of g modes. Compute initial models of massive stars including parameterized baroclinic AM transport. Upgrade GYRE to include rotational effects, and run improved simulations of AM transport by heat-driven modes. Run Dedalus simulations of convective boundary mixing.

Year 2: Perform Dedalus simulations of baroclinic instabilities to calibrate parameterized AM transport. Also perform initial Dedalus simulations of convective wave excitation. Consider effects of multiple pulsation modes and other instabilities on AM transport by heat-driven modes. Provide theoretical interpretation for initial TESS asteroseismic results. Perform 1D simulations of RSG pulsations.

Year 3: Complete Dedalus simulations of convective wave excitation and chemical mixing in different evolutionary stages. Apply theoretical models to TESS/Gaia observations, including convectively excited waves, measurements of convective boundary mixing, and internal rotation rates. Compute models of compact object rotation including baroclinic instabilities and IGW AM transport, and compare with LIGO data.

\emph{The following text is problematic because we have budgeted significantly for travel, including accommodation costs. Suggest the following paragraph as alternative text}. We will hold bi-yearly meetings including each investigator and their students/postdocs working on the proposed research. These meetings focus on an informal discussion of ideas with the goal of identifying, developing, and completing the most interesting and important research projects. We actively encourage collaboration and a free exchange of ideas. The PIs' prior collaboration has stimulated a tremendous number of new ideas and have led to a few high impact new projects. In addition to speeding scientic progress and maintaining key connections among the collaboration nodes, we will nourish and sustain the theoretical core of the discipline and train young scholars in our novel collaborative environment. UCSB offers convenient convening facilities that will be available for the collaborative meetings proposed here. In particular, UCSB offers the Sedgwick Reserve for both meetings and accommodations, which was already used by this group for our initial planning meeting. In addition, Co-PI Bildsten as KITP Director can arrange for use of both Kohn Hall and the Munger Residence for collaborative gatherings when space allows.

We will hold bi-yearly meetings including each investigator and their students/postdocs working on the proposed research. These meetings 3-day meetings will focus on an informal discussion of ideas with the goal of identifying, developing, and completing the most interesting and important research projects. These will be in neutral, isolated, locations conducive to prolonged interactions, but within driving distance from one of our sites; our previous experience has showed this format to be \emph{extremely} successful in allowing participants to engage with each another without the distractions of day to 

We actively encourage collaboration and a free exchange of ideas. The PIs' prior collaboration has stimulated a tremendous number of new ideas and have led to a few high impact new projects. In addition to speeding scientific progress and maintaining key connections among the collaboration nodes, we will nourish and sustain the theoretical core of the discipline and train young scholars in our novel collaborative environment. 