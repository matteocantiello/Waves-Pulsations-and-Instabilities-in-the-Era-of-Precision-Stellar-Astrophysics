{\color{purple}
\subsubsection{Convective Boundary Mixing}
}

Convective boundary mixing (CBM) is a major uncertainty in stellar evolution. It is especially important when nuclear burning occurs in a convection zone, as CBM can mix fresh fuel into the burning region, changing the time spent in different burning stages. The effects of CBM can be measured in massive main sequence stars by observations of eclipsing binary systems (e.g., Stancliffe et al, 2015), and via asteroseismology (e.g., Aerts et al, 2003).

A key theoretical question is how to parameterize CBM. The ``step overshoot'' parameterization simply extends the convection zone by some fraction of the local pressure scale height (cite). Another idea is ``exponential overshoot,'' where the convective mixing efficiency decreases exponentially over a lengthscale (Freytag et al 1996). Others have shown that certain quantities such as the convective velocity or pdfs of the heat flux decrease exponentially away from a convection zone (cite).

To study CBM directly, Lecoanet et al (2016) studied the evolution of a chemical species in three-dimensional Dedalus simulations.  These Cartesian simulations contained a convection zone with an adjacent radiative zone. The horizontal average of the chemical species evolved according to a diffusion equation, where the diffusivity varied with distance from the boundary of the convection zone. Specifically, the diffusivity decreases like a Gaussian outside the convection zone, i.e., much faster than exponentially.

This shows that CBM can accurately modeled as a spatially varying diffusivity, and this diffusivity can be determined directly from three-dimensional simulations. \textbf{We will run a suite of three-dimensional convection simulations to measure the convective diffusivity as a function of stellar parameters, e.g., size of convection zone, convective Mach number, properties of the radiative zone.} We will implement these radial diffusivity profiles in MESA by interpolating our simulation results to the convective parameters of a stellar model at each stage of its evolution. Thus, the CBM will change self-consistently as the star evolves.


