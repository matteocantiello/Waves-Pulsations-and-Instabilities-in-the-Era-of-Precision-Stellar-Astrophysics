{\color{purple}
\subsubsection{Convective Boundary Mixing}
}

Convective boundary mixing (CBM) is a major uncertainty in stellar evolution. It is especially important when nuclear burning occurs in a convection zone, as CBM can mix fresh fuel into the burning region, changing the time spent in different burning stages. The effects of CBM can be measured in massive main sequence stars by observations of eclipsing binary systems (e.g., Stancliffe et al, 2015), and via asteroseismology (e.g., Aerts et al, 2003).

A key theoretical question is how to parameterize CBM. The ``step overshoot'' parameterization simply extends the convection zone by some fraction of the local pressure scale height (cite). Another idea is ``exponential overshoot,'' where the convective mixing efficiency decreases exponentially over a lengthscale (Freytag et al 1996).

To address this question, Lecoanet et al (2016) studied the evolution of a chemical species in three-dimensional Dedalus simulations.  These Cartesian simulations contained a convection zone with an adjacent radiative zone. They found that the horizontal average of the chemical species evolved according to a diffusion equation, with a diffusivity which varied with distance from the boundary of the convection zone. Specifically, they found the diffusivity decreases like a gaussian 

Lecoanet et al 2016 showed, for the first time, that the evolution of a chemical species in a self-consistent 

