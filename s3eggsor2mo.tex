\subsubsection{Convective Boundary Mixing}

Convective boundary mixing (CBM) is a major uncertainty in stellar evolution. It is especially important when nuclear burning occurs in a convection zone, as CBM can mix fresh fuel into the burning region, changing the time spent in different burning stages. The effects of CBM can be measured through observations of eclipsing binary systems \citep[e.g.,][]{Stancliffe_2015,claret:16,Valle_2016}, and via asteroseismology \citep[e.g.,][]{Constantino:2015,deheuvels:16,Ghasemi_2016}. A key theoretical question is how to parameterize CBM. The ``step overshoot'' parameterization simply extends the convection zone mixing by some fraction of the local pressure scale height \citep[e.g.,][]{Shaviv_1973}, while in ``exponential overshoot'' the convective mixing efficiency decreases exponentially over a parameterized lengthscale \citep{Freytag1996}. {\colorOthers have shown that certain quantities such as the convective velocity or heat flux variations decrease exponentially away from a convection zone \citep{Pratt_2017}.